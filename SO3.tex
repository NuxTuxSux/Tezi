\newcommand{\SOtre}{SO_3(\mathbb{R})}
\newcommand{\Rtre}{\mathbb{R}^3}

\section{Classificazione dei sottoruppi finiti di $\SOtre$}
Per visualizzare in modo pi\`u concreto alcuni gruppi di automorfismi di politopi regolari, possiamo passare al caso di politopi convessi,
in cui abbiamo una naturale realizzazione di tali gruppi come automorfismi dello spazio euclideo. Ossia, considerando una versione concreta
del politopo che sar\`a quindi un politopo convesso, il suo gruppo di simmetrie sar\`a un sottogruppo del gruppo delle isometrie di 
$\mathbb{R}^n$, cio\`e il gruppo $O_n(\mathbb{R})$. Nel presente capitolo, motivati dal calcolo dei gruppi di simmetrie dei politopi astratti,
guarderemo alle simmetrie dirette, le rotazioni, ossia quelle che preservano l'orientazione, nel caso tridimensionale. Questo \`e chiaramente
l'esempio primo e fondante della teoria, emergendo naturalmente dai cosiddetti \emph{solidi platonici}.
In questo capitolo studieremo quindi il caso tridimensionale da un punto di vista algebrico procedendo, quindi, con la caratterizzazione dei
sottogruppi finiti di $SO_3=\SOtre$.\\
Per tali sottogruppi, considereremo le loro azioni sullo spazio euclideo, in particolare sul sotto $SO_3-$sottoinsieme
$S^2=\left\{v\in\Rtre\mid\|v\|=1\right\}$ della $2-$sfera unitaria.\\
Sia $G\leq SO_3$ finito, consideriamo l'azione $G \curvearrowright S^2$ ponendo
\begin{equation*}
X=\left\{x\in S^2\mid G_x\neq\left\{1\right\}\right\}
\end{equation*}
l'insieme dei punti che vengono fissati da qualche rotazione non identica di $G$.\\
Osserviamo innanzitutto che $\left|X\right|\leq 2(\left|G\right|-1)$, in particolare $X$ \`e un $G$~-~sottoinsieme finito di $S^2$.\\
Infatti ogni rotazione non identica di $G$ fissa esattamente due punti e
\begin{equation*}
X=\bigcup_{g\in G\setminus\left\{1\right\}}Fix_{S^2}g
\end{equation*}
quindi
\begin{equation*}
\left|X\right|\leq\sum_{g\in G\setminus\left\{1\right\}}\left|Fix_{S^2}g\right|=2(\left|G\right|-1)
\end{equation*}
Osserviamo, inoltre, che $\forall g\in G, \forall x\in X\quad G_{gx}=G_x^{g^{-1}}$, pertanto $X$ \`e effettivamente un
$G$-sottoinsieme di $S^2$.
\begin{oss}
Se $G$ fissa (almeno) un punto di $S^2$ allora \`e ciclico.
\end{oss}
Infatti, $\forall x \in S^2\quad (SO_3)_x\cong SO_2\cong\mathbb{S}$ quindi $G\hookrightarrow\mathbb{S}$ pertanto $G\cong C_n$ dove
$n=\left|G\right|$.\\
A meno di coniugio per una rotazione che mandi il primo vettore della base canonica dell'asse comune $\left<x\right>$ di $G$, gli elementi
di $G$ hanno matrici (della forma di):
\begin{equation*}
\left(\begin{array}{ccc}
1 & 0 & 0\\
0 & \cos{\frac{2\pi}{n}} & -\sin{\frac{2\pi}{n}}\\
0 & \sin{\frac{2\pi}{n}} & \cos{\frac{2\pi}{n}}
\end{array}\right)^n\quad \text{per } k=0,\dots,n-1.
\end{equation*}
\subsection{Studio di $G$ mediante l'azione di $G\curvearrowright X$}
Sia $X=\left\{x\in S^2 \mid G_x\neq\left\{1\right\}\right\}$ come sopra. Conteggiando gli elementi di $X$ con le molteplicit\`a date da
\begin{equation*}
m(x)=\left|G_x\right|
\end{equation*}
si ha:
\begin{equation*}
\sum_{x\in X}(\left|G_x\right|-1)=\sum_{x\in X}(m(x)-1)=\\
\sum_{g\in G\setminus\left\{1\right\}}2=2(\left|G\right|-1).
\end{equation*}
Sia quindi $Y$ una scelta di rappresentanti per $X/G=\left\{x^G\mid x\in X\right\}$, notando che se $x_1^G=x_2^G$ allora
$m(x_1)=m(x_2)$, avremo
\begin{equation*}
2(\left| G\right|-1)=\sum_{x\in X}(m(x)-1)=\sum_{y\in Y}\left|y^G\right|(m(y)-1)=\sum_{y\in Y}\frac{\left|G\right|}{m(y)}(m(y)-1).
\end{equation*}
Da cui, dividendo per la cardinalit\`a di $G$, si ottiene
\begin{equation} \label{eq:1}
\sum_{y\in Y}\frac{m(y)-1}{m(y)}=2-\frac{2}{\left|G\right|}
\end{equation}
Ricordiamo che $y\in Y\subseteq X$ implica $G_y\neq\left\{1\right\}$, cio\`e $m(y)\geq 2$, quindi, nella \ref{eq:1},
\begin{equation*}
2-\frac{2}{\left| G\right|}=\sum_{y\in Y}(1-\frac{1}{m(y)})\geq\left|Y\right|(1-\frac{1}{2})=\frac{\left|Y\right|}{2}
\end{equation*}
pertanto
\begin{equation*}
\frac{\left|Y\right|}{2}\leq 2-\frac{2}{\left|G\right|}<2
\end{equation*}
e $\left|Y\right|<4$, cio\`e $\left|Y\right|\leq 3$.\\
Pertanto $X$ si decompone, sotto l'azione di $G$, in al pi\`u tre orbite.\\
\begin{oss}
Notiamo che, in effetti, non pu\`o verificarsi il caso che ci sia una sola orbita (ossia l'azione $G\curvearrowright X$ non \`e transitiva).
Infatti, se fosse $\left|Y\right|=1$, avremmo (essendo $\left|G\right|>1$) dalla \ref{eq:1}:
\begin{equation*}
1>1-\frac{1}{m(y)}=2-\frac{2}{\left|G\right|}\geq 1\qquad\text{dove}\quad\left\{y\right\}=Y
\end{equation*}
\end{oss}

Quindi $X$ si decompone come unione di due o tre orbite sotto l'azione del gruppo $G$\\
Vedremo che si presenteranno entrambi i casi e questi daranno luogo, enl primo caso alla famiglia dei gruppi ciclici diniti $C_n$
con $n\in\mathbb{N}^+$ e alla famiglia dei gruppi diedrali $D_n$ con $n\in\mathbb{N}^+$ e, nel secondo a tre altri sottogruppi
(che si riveleranno essere i pi\`u geometricamente interessanti ai nostri fini).
\subsection{Caso $\left|Y\right|=2$}
Studiamo il caso in cui $X$ si decompone in due orbite sotto l'azione di $G$.\\
Sia $Y=\left\{y_1,y_2\right\}$ e poniamo $m=m(y_1)$, $n=m(y_2)$. Allora la \ref{eq:1} diventa:
\begin{equation*}
1-\frac{1}{m}+1-\frac{1}{n}=2-\frac{2}{\left|G\right|},\qquad\text{cio\`e}
\end{equation*}
\begin{equation*}
\frac{1}{m}+\frac{1}{n}=\frac{2}{\left|G\right|}\qquad\text{da cui}
\end{equation*}
\begin{equation*}
\frac{\left|G\right|}{m}+\frac{\left|G\right|}{n}=2
\end{equation*}
ed, essendo $m=\left|G_{y_1}\right|$ e $n=\left|G_{y_2}\right|$,
si ha che $\frac{\left|G\right|}{m},\frac{\left|G\right|}{n}\in\mathbb{N}^+$.
Quindi le uniche soluzioni sono per $m=n=\left|G\right|$.\\
Pertanto $G=G_{y_1}=G_{y_2}$ agisce banalmente su $y_1$ e $y_2$ ($y_i^G=\left\{y_i\right\}$ per $i=1,2$) e
$Y=Fix_{S^2}g\quad\forall g\in G$. Cio\`e gli elementi di $G$ sono tutte rotazioni con asse $\left<y_1\right>=\left<y_2\right>$.\\
Quindi in questo caso si ottiene la famiglia infinita dei gruppi ciclici (finiti) che risultano coniugati nel rispetto dell'ordine
(per la transitivit\`a dell'azione $SO_3\curvearrowright S^2$).

\subsection{Caso $\left|Y\right|=3$}
Questo caso risulter\`a pi\`u interessante, racchiudendo la famiglia dei gruppi diedrali e altri tre gruppi.
Identificheremo dei sottocasi derivanti dall'equazione \ref{eq:1}.\\
Sia
\begin{equation*}
Y=\left\{y_1,y_2,y_3\right\}
\end{equation*}
Posti $m=m(y_1), n=m(y_2), k=m(y_3)$ la \ref{eq:1} diventa:
\begin{equation*}
1-\frac{1}{m}+1-\frac{1}{n}+1-\frac{1}{k}=2-\frac{2}{\left|G\right|}\qquad\text{cio\`e:}
\end{equation*}
\begin{equation} \label{eq:2}
\frac{1}{m}+\frac{1}{n}+\frac{1}{k}=1+\frac{2}{\left|G\right|}
\end{equation}
Cerchiamo le soluzioni $(m,n,k)$ supposte $m\leq n\leq k$ di \ref{eq:2}.\\
Notiamo d'apprima che $m=2$, infatti $m,n,k\geq 2$ e, se avessimo $m,n,k\geq 3$, allora otterremmo
\begin{equation*}
\frac{1}{m}+\frac{1}{n}+\frac{1}{k}\leq\frac{1}{3}+\frac{1}{3}+\frac{1}{3}=1\quad\text{contro la \ref{eq:2}}
\end{equation*}
Pertanto $m=2$ e riscriviamo la \ref{eq:2} in
\begin{equation*}
\frac{1}{2}+\frac{1}{n}+\frac{1}{k}=1+\frac{2}{\left|G\right|}\qquad\text{cio\`e:}
\end{equation*}
\begin{equation} \label{eq:3}
\frac{1}{n}+\frac{1}{k}=\frac{1}{2}+\frac{2}{\left|G\right|}
\end{equation}
Notiamo adesso che non pu\`o essere $n,k\geq 4$, altrimenti avremmo
\begin{equation*}
\frac{1}{n}+\frac{1}{k}\leq\frac{1}{4}+\frac{1}{4}=\frac{1}{2}\qquad\text{contro la \ref{eq:3}}
\end{equation*}
Allora $n$ pu\`o assumere solamente due valori: $2$ o $3$.

\begin{oss} \label{oss:1}
Siano $\sigma,\rho\in SO_3\setminus\left\{1\right\}$ tali che $\sigma\in N_{SO_3}(\left<\rho\right>)\setminus C_{SO_3}(\left<\rho\right>)$. Allora $\sigma$ \`e un'involuzione con asse perpendicolare all'asse di $\rho$ e $\left<\sigma,\rho\right>\cong D_k$, dove $k=o(\rho)$.
\end{oss}
\begin{proof}
Siano $\left\{\pm y\right\}=Fix_{S^2}\rho$, allora $\left<\sigma\right>\curvearrowright Fix_{S^2}\rho$ quindi
\begin{gather*}
\sigma^{-1}\rho\sigma(y)=\rho^\sigma(y)=y\qquad\text{cio\`e}\\
\rho\sigma(y)=\sigma(y)\quad\text{e}\quad \sigma(y)\in Fix_{S^2}\rho\\
(\text{quindi anche}\quad \sigma(-y)\in Fix_{S^2}\rho)
\end{gather*}
Se avessimo $\sigma(y)=y$ allora $\sigma\in C_{SO_3}(\left<\rho\right>)$ perch\'e coassiali, quindi $\sigma(\pm y)~=~\mp~y$, cio\`e
$y\in Aut_{-1}\sigma\neq\left\{0\right\}$ e
\begin{equation*}
\mathbb{R}^3=Aut_1\sigma\overset{\scriptscriptstyle{\bot}}\oplus Aut_{-1}\sigma
\end{equation*}
Dove $Aut_1\sigma$ \`e l'asse di $\sigma$ e $Aut_{-1}=(Aut_1\sigma)^\bot$ \`e il suo piano ortogonale, sul quale $\sigma$ agisce antipodalmente.\\
Se $x\in Aut_1\sigma\setminus\left\{0\right\}$ e $\mathcal{B}=\left(y,x,z\right)$ \`e una base ortonormale relativa a tale decomposizione,
posti $R={}_\mathcal{B}\!\left[\rho\right]_\mathcal{B}$	e $S={}_\mathcal{B}\!\left[\sigma\right]_\mathcal{B}$,		% 	QUESTO FA SCHIFO
si ha 
\begin{equation*}
S=\left(\begin{array}{ccc}
-1 & 0 & 0\\
0 & 1 & 0\\
0 & 0 & -1
\end{array}\right)
\qquad\text{in particolare}\quad S^2=1\quad\text{e}
\end{equation*}

\begin{equation*}
R=\left(\begin{array}{ccc}
1 & 0 & 0\\
0 & \cos{\theta} & -\sin{\theta}\\
0 & \sin{\theta} & \cos{\theta}
\end{array}\right)
\qquad\text{se }\theta\text{ \`e l'angolo di }\rho\text{ e}
\end{equation*}


\begin{equation*}
\begin{split}
R^S & =\left(\begin{array}{ccc}
-1 & 0 & 0\\
0 & 1 & 0\\
0 & 0 & -1
\end{array}\right)
\left(\begin{array}{ccc}
1 & 0 & 0\\
0 & \cos{\theta} & -\sin{\theta}\\
0 & \sin{\theta} & \cos{\theta}
\end{array}\right)
\left(\begin{array}{ccc}
-1 & 0 & 0\\
0 & 1 & 0\\
0 & 0 & -1
\end{array}\right)=\\
& =\left(\begin{array}{ccc}
1 & 0 & 0\\
0 & \cos{\theta} & \sin{\theta}\\
0 & -\sin{\theta} & \cos{\theta}
\end{array}\right)=R^{-1}
\end{split}
\end{equation*}

Riassumendo, $R^k=S^2=1$ e $R^S=R^{-1}$, pertanto $G=\left<R,S\right>\cong D_k$.
\end{proof}
Proseguiamo quindi con lo studio del gruppo $G$ nel
\subsection{Caso ($m=2$) $n=2$}
Dalla formula \ref{eq:3} notiamo che $k=\frac{\left|G\right|}{2}=\left|G_{y_3}\right|$. Pertanto $G_{y_3}\trianglelefteq G$,
quindi sia $\sigma\in G\setminus G_{y_3}$ tale che $\sigma$ non centralizza $G_{y_3}$, altrimenti fisserebbe $y_3$ e possiamo
utilizzare l'osservazione \ref{oss:1} per concludere che $\sigma$ \`e un'involuzione e che $G\cong D_k$ \`e il $k$-esimo gruppo
diedrale.
Inoltre una sua presentazione come sottogruppo di $SO_3$ \`e data nell'osservazione \ref{oss:1} stessa.

Continuiamo adesso con lo studio del terzo (sotto)caso che corrisponder\`a ai gruppi di simmetria dei poliedri regolari
di $\mathbb{R}^3$ (i \emph{solidi platonici}).
Facciamo una considerazione che ci sar\`a a tal fine utile.
\begin{oss} \label{oss:2}
Sia $G\leq SO_3$ e $X$ un $G$-sottoinsieme di $S^2$ con $\left|X\right|\geq 3$. Allora l'azione di $G$ su $X$ \`e fedele.
\end{oss}
\begin{proof}
Se $G\overset{\alpha}\longrightarrow S_X$ \`e la restrizione dell'azione, sia $g\in Ker\alpha$. Allora
$\left|Fix_{S^2}g\right|=\left|X\right|\geq 3$ quindi, essendo $g\in SO_3$ abbiamo $g=1$
\end{proof}

Riscriviamo adesso la formula \ref{eq:3} nell'ultimo caso, per cui vale $n=3$:
\begin{equation} \label{eq:4}
\frac{1}{k} = \frac{1}{6} + \frac{2}{\left|G\right|}
\end{equation}
Da questa notiamo che $\frac{1}{k}>\frac{1}{6}$, quindi $k$ pu\`o assumere i tre valori $3,4,5 (k~\geq~n~=~3)$.

Studiamo singolarmente i tre (sotto)casi, nei quali vale:
\begin{equation} \label{eq:5}
\left|G\right| = \frac{12k}{6-k}
\end{equation}
\subsection{Caso $(m=2,n=3)\ k=3$}
Innanzitutto notiamo che $\left|G\right|=12$ e vedremo che questo determiner\`a univocamente la struttura di $G$.
Per capire di che gruppo si tratti, basta notare che
\begin{equation*}
\left|y_3^G\right| = \left[G:G_{y_3}\right]=\frac{\left|G\right|}{m(y_3)}=\frac{12}{3}=4\quad\text{(ricordiamo che }k=m(y_3)\text{)}
\end{equation*}

Pertanto, per l'osservazione \ref{oss:2}, l'azione di $G$ su $y_3^G$ \`e fedele, quindi abbiamo un'immersione di $G$ in $S_4$.
Per concludere che l'immagine di questa immersione \`e effettivamente $A_4$ basta notare che $G$ possiede otto elementi di ordine $3$
le cui immagini in $S_4$ devono essere gli otto $3$-cicli, che generano $A_4$.

Concludiamo che 
\begin{equation*}
G\cong A_4
\end{equation*}

Nel seguito considereremo, oltre alle azioni del gruppo $G$ su $X$ e sulle relative orbite, anche quella che ne deriva
sugli stabilizzatori.

Questo permette di semplificare lo studio di $G$ tramite l'azione du un insieme (in generale) pi\`u piccolo.
Inoltre quest'azione ha un esplicito significato geometrico, possiamo infatti immaginare gli stabilizzatori come una
``versione algebrica'' degli assi degli elementi di $G$.
Ad ogni asse possiamo infatti far corrispondere lo stabilizzatore della sua intersezione con $S^2$ e per il viceversa
abbiamo visto che gli elementi non identici di uno stabilizzatore condividono l'asse.

\begin{oss} \label{oss:3}
Siano $G\leq SO_3$ che agisce su $X=\left\{v\in S^2\mid G_v\neq\left\{1\right\}\right\}$ e $x, y\in X$. Allora
\begin{equation*}
G_x = G_y\quad \Longleftrightarrow\quad x=\pm y.
\end{equation*}
\begin{proof}
Sia $g\in G_x\setminus\left\{1\right\}$ da cui $Fix_{S^2}g = \left\{\pm x\right\}$ ma anche
$g\in G_y\setminus\left\{1\right\}$ da cui $Fix_{S^2}g = \left\{\pm y\right\}$ e quindi la tesi
\end{proof}
\end{oss}

\begin{fatto} \label{fatto:1}
Sia $H$ un gruppo agente su $T$. Allora $H$ agisce per coniugio \emph{sui suoi stabilizzatori}, infatti
\begin{equation} \label{azStab}
H_x^g = H_{g^{-1}x}\qquad\forall g\in H,\forall x\in T.
\end{equation}
Pertanto, se l'azione di $H$ su $T$ \`e transitiva, lo \`e anche quella su $Z~=~\left\{H_x\mid x\in T\right\}$.
\end{fatto}
\begin{proof}
Per la genericit\`a di $g\in H$ e $x\in T$ basta mostrare che $H_x^g\leq H_{g^{-1}x}$. Sia allora $h\in H_x$, quindi
\begin{equation*}
h^g(g^{-1}x) = g^{-1}hgg^{-1}x = g^{-1}hx = g^{-1}x
\end{equation*}
quindi $h^g\in H_{g^{-1}x}$
\end{proof}

Proseguiamo quindi con il prossimo

\subsection{Caso $(m=2, n=3)\ k=4$}
Innanzitutto, per la \ref{eq:5}, notiamo che $\left|G\right|=24$. Inoltre l'azione di $G$ su $X$ decompone quest'ultimo in tre orbite
di $12$, $8$ e $6$ elementi ciascuna:
\begin{equation*}
X = y_1^G\overset{\cdot}\cup y_2^G\overset{\cdot}\cup y_3^G
\end{equation*}
con stabilizzatori (ciclici) di cardinalit\`a rispettivamente $2=m, 3=n, 4=k$.

Per studiare $G$ individuiamo un pi\`u piccolo insieme su cui esso agisce. Scegliamo, come preannunziato,
\begin{equation*}
Z = \left\{ G_x\mid x\in y_2^G\right\}
\end{equation*}

Per il fatto \ref{fatto:1} $G$ agisce transitivamente su $Z$ per coniugio.

Determiniamo $\left|Z\right|$.
A tal fine, chiamata $i\in O_3$ l'applicazione antipodale ($i=-1$), notiamo che $G\leq O_3=C_{O_3}(i)$ ($i$ \`e scalare), pertanto
gli insiemi $iy_1^G,iy_2^G,iy_3^G,$ sono stabili per l'azione di $G$.
Quindi $\forall j\in\left\{1,2,3\right\}\quad iy_j^G$ \`e unione di $G$-orbite ma, guardando le cardinalit\`a di quest'ultima \`e
necessariamente
\begin{equation*}
iy_j^G = y_j^G
\end{equation*}
Le orbite sono quindi simmetriche rispetto all'origine.

L'osservazione \ref{oss:3} ci permette di concludere che $\left|Z\right| = 4$. Mostriamo quindi che tale azione di $G$ su $Z$
\`e \emph{fedele}.
Notiamo intanto che $G$ \`e unione dei suoi stabilizzatori rispetto alle azioni su $y_1^G,y_2^G,y_3^G$.

Questi sono sottogruppi ciclici (massimali) di rispettivamente $m=2,\ n=3,\ k=4$ elementi, tutti coniugati nel rispetto delle cardinalit\`a.
\begin{equation*}
G\setminus\left\{1\right\} = \underset{x\in X}{\overset{\cdot}\bigcup}(G_x\setminus\left\{1\right\})=
\overset{\cdot}{\underset{\substack{i=1,2,3\\g\in G}}\bigcup}(G_{y_i}^g\setminus\left\{1\right\})
\end{equation*}
Ma torniamo alla rappresentazione $G\overset\phi\longrightarrow S_Z$.
Sia $N=Ker\phi$, mostriamo che $N=\left\{1\right\}$.
Innanzitutto $N$ non contiene elementi di ordine $3$. Infatti sia $\rho\in G$ con
$o(\rho)=3$. Se $H\leq G, \left|H\right|=3$ (cio\`e se $H\in Z$) tale che $\rho\notin H$, allora $\rho\in N_G(H)\geq Ker\phi=N$
infatti, se cos\`i fosse, $\left<\rho,H\right>\leq G$ avrebbe $9$ elementi.

$N$ non contiene neanche $4$-stabilizzatori, altrimenti li conterrebbe tutti, ma questi sono $\frac{\left|y_3^G\right|}{2}=\frac{6}{2}=3$
(ricordando che $G_x=G_y \Leftrightarrow x = \pm y$ e $y_3^G=-y_3^G$) pertanto avrebbe almeno le tre coppie di generatori e l'identit\`a,
quindi almeno $7$ elementi, perci\`o $\left|N\right|\geq 8$ e $\left|G/N\right|\leq 3$ perci\`o $G/N$ non potrebbe agire transitivamente
su $Z$.

Un argomento analogo si pu\`o fare per vedere che $N$ non contiene nessuno dei $6$ $2$-stabilizzatori.
Ci resta da vedere che $N$ non contiene nessuna delle incoluzioni dei $4$-stabilizzatori. Se ne contenesse una,
infatti, le conterrebbe tutte (i $4$-stabilizzatori sono tutti coniugati e ognuno contiene un'unica involuzione).

Notiamo che - come vedremo - esiste in effetti in $G$ un sottogruppo normale (isomorfo a $K_4$, il \emph{gruppo di Klein}) costituito
da tali involuzioni a l'identit\`a.

Tuttavia se $N$ fosse costituito da queste $3$ involuzioni e l'identit\`a, $G/N$ sarebbe un gruppo di $6$ elementi agente
transitivamente su $Z$, di $4$ elementi, pertanto avremmo:
\begin{equation*}
4=\left|Z\right|\big|\left|G\right|=6,\qquad\text{assurdo.}
\end{equation*}

Pertanto $G\hookrightarrow S_Z\cong S_4$ ed, essendo $\left|G\right|=\left|S_4\right|$ si conclude che
\begin{equation*}
G\cong S_4
\end{equation*}

Studiamo quindi l'ultimo
\subsection{Caso ($m=2,\ n=3)\ k=5$}
Innanzitutto, per la \ref{eq:5}, abbiamo che $\left|G\right|=60$ e le orbite dell'azione di $G$ su $X$ hanno cardinalit\`a
\begin{equation*}
\left|y_1^G\right|=30\qquad\left|y_2^G\right|=30\quad\text{e}\quad\left|y_3^G\right|=12
\end{equation*}
Come nel caso $k=4$, tali orbite sono simmetriche rispetto all'origine.

Notiamo che tutti i sottogruppi di ordine $3$ sono coniugati e cos\`i quelle di ordine $5$ e, per quanto osservato, questi sono
rispettivamente
\begin{equation*}
\frac{\left|y_2^G\right|}{2}=10\quad\text{e}\quad\frac{\left|y_3^G\right|}{2}=6
\end{equation*}
in numero.

Inoltre vi saranno (almeno) altri $\frac{\left|y_1^G\right|}{2}=15$ $2$-stabilizzatori (involuzioni). Questi ci daranno quindi
$10\cdot 2=20$ elementi di ordine $3$ e $6\cdot 4=24$ elementi di ordine $5$. Osserviamo che $1+20+24+15=60$, pertanto le involuzioni
saranno precisamente le $15$ che generano gli altrettanti stabilizzatori relativi a $y_1^G$.

Gli unici ordini di elementi non identici in $G$ sono quindi $2$,$3$ e $5$ e i sottogruppi ciclici sono tutti coniugati nel rispetto
delle cardinalit\`a. Continuiamo lo studio di $G$ mostrando - grazie alla simmetria fra i suoi sottogruppi - che esso \`e semplice.

Sia $N\trianglelefteq G$, se $N$ contiene un elemento non identico allora, essendo questi coniugati a meno dell'ordine,
\begin{gather*}
\left|N\right|\geq\min{\left\{15,20,24\right\}}+1=16\\
\left|N\right|\in\left\{20,30,60\right\}
\end{gather*}
pertanto $10\mid\left|N\right|$ e $N$ ha un'involuzione e un elemento di ordine $5$, quindi
\begin{equation*}
\left|N\right|\geq 15+24+1=40\Rightarrow N=G.
\end{equation*}

Studiamo adesso i $2$-Sylow di $G$.

Questi sono isomorfi a $K_4$ (in quanto $G$ non possiede elementi di ordine $4$). Notiamo poi che ogni involuzione di $G$ \`e contenuta
in un tale $2$-Sylow, infatti $\forall\sigma\in G$ involuzione, scelti qualunque $1\neq\sigma_1\in Q\in Syl_2G$ esiste un $g\in G$
tale che $\sigma_1^g=\sigma$, allora $\sigma\in Q^g\in Syl_2G$.
Inoltre tale $2$-Sylow risulta essere il centralizzante dell'involuzione (pertanto esso \`e unico). Cio\`e abbiamo che
\begin{equation*}
Syl_2G = \left\{G_G(\sigma)\mid\sigma\in G\text{ involuzione}\right\}.
\end{equation*}
Quanti sono tali $2$-Sylow?

Notiamo che se $Q,Q'\in Syl_2G$ hanno intersezione non banale, diciamo
\begin{equation*}
1\neq\sigma\in Q\cap Q'
\end{equation*}
allora
\begin{equation*}
\left<Q,Q'\right>\leq C_G(\sigma)
\end{equation*}
quindi $Q=Q'$. I $2$-Sylow di $G$ sono pertanto a due a due digiunti.

Da questo concludiamo che le $15$ involuzioni di $G$ si partizionano in
\begin{equation*}
\left\{\sigma\in G\mid\sigma involuzione\right\}=\underset{Q\in Syl_2G}{\overset{\cdot}\bigcup}(Q\setminus\left\{1\right\})
\end{equation*}
Pertanto 
\begin{equation*}
\left|Syl_2G\right|=5.
\end{equation*}
Questo ci permette praticamente di determinare il gruppo $G$. Infatti $G$ agisce transitivamente sui suoi $2$-Sylow e abbiamo, quindi,
\begin{equation*}
G\overset\phi\longrightarrow S_{Syl_2G}\cong S_5
\end{equation*}
Per la transitivit\`a dell'azione abbiamo che $Ker\neq G$ quindi, per la semplicit\`a di $G$, $Ker\phi=\left\{1\right\}$ e $\phi$ risulta
un'immersione.

A questo punto concludiamo che
\begin{equation*}
G\cong A_5
\end{equation*}
Infatti indicando con $\Tilde{G}$ l'immagine di $G$ in $S_5$, abbiamo che
\begin{equation*}
[\Tilde{G}:\Tilde{G}\cap A_5]\leq 2\ \Rightarrow\ \Tilde{G}\cap A_5\trianglelefteq\Tilde{G}\ \Rightarrow\ \Tilde{G}\cap A_5=\Tilde{G}
\end{equation*}
(essendo $\Tilde{G}$ semplice e $|\Tilde{G}\cap A_5|\geq 30>1$).


