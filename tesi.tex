\documentclass[12pt,a4paper,twoside]{book}

%\usepackage{pdfx}
\usepackage[style=numeric-comp,backend=biber]{biblatex}

\usepackage{rotating}
\usepackage{graphicx} 
\usepackage[table,xcdraw]{xcolor}
\usepackage{longtable}
\usepackage{colortbl}
\usepackage{color}
\usepackage{tabularx}
\usepackage{ltxtable}
\usepackage{array}
\usepackage{eurosym}
%\usepackage[bookmarks]{hyperref}
\usepackage{amstext}
\usepackage{enumerate}
\usepackage{booktabs} 
\usepackage{verbatim}
\usepackage{url}
\usepackage[small,bf]{caption}
\usepackage[a4paper]{geometry}
\usepackage{graphicx}
\usepackage{url}
\usepackage{float}
\usepackage{cancel}
\usepackage{bm}
\usepackage{amssymb,amsmath,amsthm,amsfonts}
\usepackage{multirow}
\usepackage{tikz}
\usepackage{listings}
\usepackage{url}

\usepackage{setspace}
\usepackage{epigraph}
\setlength{\epigraphrule}{0pt}

\usetikzlibrary{arrows}
\usetikzlibrary{patterns}
\usetikzlibrary{decorations.pathmorphing}

\usepackage[utf8]{inputenc} %allows accents without escaping
\usepackage[T1]{fontenc}
\usepackage{listings} 

% super utile, è il modo corretto di fare i ... nelle formule secondo Knuth
\newcommand{\lst}[2]{${#1}_0$,~${#1}_1$, $\dots\,$,~${#1}_{#2-1}$}

\newcommand{\argmax}[1]{\underset{#1}{\operatorname{arg}\,\operatorname{max}}\;}

\newtheorem{theorem}{Theorem}
\newtheorem{corollary}{Corollary}[theorem]
\newtheorem{lemma}[theorem]{Lemma}
\newtheorem{definition}{Definition}[section]
\newtheorem*{theorem*}{Theorem}
\newtheorem*{lemma*}{Lemma}
\newtheorem*{corollary*}{Corollary}




\newcommand{\Rn}{\mathbb{R}^n}
\newcommand{\p}{\mathcal{P}}
\newtheorem{defin}{Definizione}
\newtheorem{lem}{Lemma}
\newtheorem{teo}{Teorema}
\newtheorem{prop}{Proposizione}




\newcommand{\frameit}[2]{
        \begin{center}
{\color{red} \framebox[0.9\textwidth][l]{
                \begin{minipage}{0.85\textwidth}
{\color{red}\bf #1}: {\color{black}#2}
                \end{minipage}
}\\
}
        \end{center}
}


% abilitare hyperref solo alla fi.e., quando si vuole il documento pronto.
%
%\usepackage[pdftex,hyperindex]{hyperref}
%\addtocontents{toc}{\protect\setstretch{1.5}}
%\def\addvspace#1{}
%\hypersetup{
	%   bookmarks=false,
	%	colorlinks=true,
	%	urlcolor=blue,
%	pdfborder = {0 0 0},
%	pdfauthor = {Alessandro Luongo},
%	pdftitle = {},
%	pdflang={it},
%	pdfcreator = {LaTeX, TeXmaker, pdfLaTeX, Hyperref}
%}

\setlength{\captionmargin}{20pt}
\setcounter{tocdepth}{3}

\bibliography{tesi}

\begin{document}
%\input{FrontPage}

\tableofcontents %indice


\chapter{Introduction}
\setlength{\epigraphwidth}{.95\textwidth}
\begin{epigraphs}
\qitem{Da quella notte\\
\`e entrato il gatto\\
con una testa in bocca\\
e me l'ha messa\\
tra i miei circuiti\\
ed ha ripreso vita\\
ed \`e da allora\\
che ho la coscienza\\
perch\`e di notte sogno\\
di quella notte\\
che \`e entrato il gatto\\
con una testa in bocca.}
{---\textsc{IL SOGNO DEL COMPUTER} --- Musica Per Bambini (Dei Nuovi Animali)}
\end{epigraphs}
\newpage
\label{ch:introduction}
%\input{Introduction}
%\input{FormalismNotation}




\chapter{Politopi astratti}
In questo capitolo parleremo dei politopi astratti, un'astrazione assiomatica della struttura combinatoria dei politopi (concreti) di $\Rn$. Essi incapsulano la
semplice struttura combinatoria delle facce di un politopo, essendo questa gi\`a molto ricca di informazioni circa il politopo stesso. Cominciamo con alcune 
definizioni.
\subsection{Definizioni}
Un insieme parzialmente ordinato $(\p,\leq)$ si dice \emph{politopo astratto di rango} $n$ (dove $rank\p:=n\in\mathbb{N})$ se soddisfa le propriet\`a $(P1)-(P4)$ scritte 
di seguito. Per una migliore comprensione delle propriet\`a in questione introduciamo la terminologia di base. Gli elementi di $\p$ verranno chiamati \emph{facce},
due facce $F,G\in\p$ verranno dette \emph{incidenti} se $F\leq G$ o $G\leq F$. Ricordiamo inoltre che una \emph{catena} di un insieme ordinato \`e un suo 
sottoinsieme totalmente ordinato. Nel caso dei politopi chiameremo \emph{lunghezza} di una catena $\Omega$ il numero $\left|\Omega\right|-1$ e, vista
l'importanza delle catene massimali, chiameremo quest'ultime \emph{bandiere}.
Denoteremo con $\mathcal{F}(\p)$ l'insieme di tali bandiere. Ovviamente ogni catena di $\p$ sar\`a contenuta in un elemento di $\mathcal{F}(\p)$.
Enunciamo i primi due assiomi di politopo astratto. Sia $\p$ un insieme ordinato.
\begin{itemize}
\item (P1) $\p$ contiene un minimo e un massimo, che chiamiamo $F_{-1}$ e $F_d$ rispettivamente.
\item (P2) Ogni bandiera di $\p$ ha lunghezza $n+1$ (contiene quindi $n+2$ elementi)
\end{itemize}
Per enunciare i restanti assiomi di politopo astratto abbiamo bisogno di definire le sue \emph{sezioni}.
\begin{defin}
Date $F,G\in\p$ chiameremo sezione di $\p$ relativa a $F$ e $G$ il suo sottoinsieme
\begin{equation*}
G/F=\left\{H\in\p\mid F\leq H\leq G\right\}
\end{equation*}
Notiamo che una sezione di un poset che soddisfa $(P1)-(P2)$ soddisfa essa stessa $(P1)-(P2)$. Mostreremo infatti che una sezione di un politopo astratto
continuer\`a ad essere un politopo astratto. Diremo inoltre che una sezione \`e propria se questa \`e distinta da $\p$.
\end{defin}
\par
Gi\`a con queste due propriet\`a osserviamo che \`e possibile estendere la nozione di \emph{rango} da $\p$ alle sue facce.
Se $F\in\p$ chiameremo $rankF:=rank(F/F_{-1})$ \emph{rango} della faccia $F$. Diremo inoltre che $F$ \`e una $i$-faccia.
Notiamo che $rank(F_{-1})=-1$ e $rank(F_n)=n$ queste facce sono ovviamente le uniche di $\p$ con tali ranghi e verranno dette
facce improprie di $\p$.
Definiamo inoltre
\begin{equation*}
\p_i=\left\{F\in\p\mid rankF=i\right\}
\end{equation*}
Chiameremo le facce di $\p$ di rango $0$ \emph{vertici} di $\p$, quelle di rango $1$ \emph{vertici} e quelle di rango $n-1$ \emph{faccette}. Data 
una qualunque faccia $F\in\p$ diremo che la sezione $P/F$ \`e la sua \emph{co-faccia}.\\
Notiamo che se $F,G\in\p$ con $F\leq G$ allora
\begin{equation*}
rank(G/F)=rankG-rankF-1
\end{equation*}
In particolare $rank(F/F_{-1})=rankF$ e $rank(P/F)=n-rankF-1$. Chiameremo quest'ultimo \emph{co-rango} di $F$.
Parlando di una catena $\Omega$ di $\p$ chiameremo \emph{tipo} di $\Omega$ l'insieme ordinato dei ranghi dei suoi elementi e scrivendo 
$\Omega=\left\{F_{i_1},\dots,F_{i_l}\right\}$ assumeremo implicitamente che $rankF_{i_t}=i_t$, che $i_1<\cdots<i_l$ e, quindi,
che $\Omega$ sia di tipo $(i_1,\dots,i_l)$.
Talvolta ometteremo nell'enumerazione di una catena le sue facce improprie.\\
A questo punto possiamo introdurre il concetto di \emph{connessione}.
\begin{defin}
Un insieme parzialmente ordinato $\p$ che gode delle propriet\`a (P1)-(P2) si dir\`a \emph{connesso} se
\begin{equation*}
\forall F,G\in\p\quad\exists F=:H_0,\cdots,H_l:=G\in\p\qquad \text{tali che }H_i\text{ e }H_{i+1}\text{ sono incidenti}
\end{equation*}
Considereremo per definizione $\p$ connesso nel caso $rank\p=1$.
\end{defin}
Arriviamo finalmente all'enunciazione del terzo assioma definendo la connessione forte di un insieme parzialmente ordinato.
\begin{defin}
Sia $\p$ un insieme parzialmente ordinato che gode delle propriet\`a (P1)-(P2). Diremo che $\p$ \`e \emph{fortemente connesso} se
\`e connessa ogni sua sezione.
\end{defin}
Come preannunciato il prossimo e terzo assioma di politopo astratto \`e
\begin{itemize}
\item (P3) $\p$ \`e fortemente connesso
\end{itemize}
Nel seguito utilizzeremo una variante di (P3) che fa uso delle bandiere del politopo. A tal fine diamo le seguenti
\begin{defin}
Sia $\p$ un insieme parzialmente ordinato che soddisfa (P1)-(P2). Due bandiere $\Phi,\Psi\in\mathcal{F}(\p)$ si dicono \emph{adiacenti}
se differiscono per un solo elemento. Se $\left\{F\right\}=\Phi\cap\Psi$ e $i=rankF$ diremo anche che $\Phi$ e $\Psi$ sono $i$-adiacenti.\\
\end{defin}
\begin{defin}
Sia $\p$ un insieme parzialmente ordinato che soddisfa (P1)-(P2). Allora diremo che $\p$ \`e connesso per bandiere se $\forall\Phi,\Psi\in\mathcal{F}(\p)$
esistono $\Phi_i\in\mathcal{F}(\p)$ per $i=0,\dots,k$ tali che
\begin{gather*}
\Phi_0=\Phi\\
\Phi_k=\Psi\\
\Phi_i\text{ e }\Phi_{i+1}\text{ sono adiacenti}
\end{gather*}
\end{defin}
In modo analogo a quanto visto per la connessione definiamo la connessione forte per bandiere.
\begin{defin}
Se $\p$ \`e un insieme parzialmente ordinato che soddisfa (P1)-(P2) allora diremo che $\p$ \`e \emph{fortemente connesso per bandiere} se
ogni sua sezione \`e connessa per bandiere
\end{defin}

\begin{prop}
Sia $\p$ un insieme parzialmente ordinato che soddisfa (P1)-(P2). Allora $\p$ \`e fortemente connesso se e solo se \`e fortemente
connesso per bandiere.
\end{prop}
\begin{proof}

\end{proof}


\chapter{Conclusioni}
Qua ci stanno le conclusioni.

\end{document}