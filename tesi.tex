\documentclass[12pt,a4paper,twoside]{book}

%\usepackage{pdfx}
\usepackage[style=numeric-comp,backend=biber]{biblatex}

\usepackage{rotating}
\usepackage{graphicx} 
\usepackage[table,xcdraw]{xcolor}
\usepackage{longtable}
\usepackage{colortbl}
\usepackage{color}
\usepackage{tabularx}
\usepackage{ltxtable}
\usepackage{array}
\usepackage{eurosym}
%\usepackage[bookmarks]{hyperref}
\usepackage{amstext}
\usepackage{enumerate}
\usepackage{booktabs} 
\usepackage{verbatim}
\usepackage{url}
\usepackage[small,bf]{caption}
\usepackage[a4paper]{geometry}
\usepackage{graphicx}
\usepackage{url}
\usepackage{float}
\usepackage{cancel}
\usepackage{bm}
\usepackage{amssymb,amsmath,amsthm,amsfonts}
\usepackage{multirow}
\usepackage{tikz}
\usepackage{listings}
\usepackage{url}

\usepackage{setspace}
\usepackage{epigraph}
\setlength{\epigraphrule}{0pt}

\usetikzlibrary{arrows}
\usetikzlibrary{patterns}
\usetikzlibrary{decorations.pathmorphing}

\usepackage[utf8]{inputenc} %allows accents without escaping
\usepackage[T1]{fontenc}
\usepackage{listings} 

% super utile, è il modo corretto di fare i ... nelle formule secondo Knuth
\newcommand{\lst}[2]{${#1}_0$,~${#1}_1$, $\dots\,$,~${#1}_{#2-1}$}

\newcommand{\argmax}[1]{\underset{#1}{\operatorname{arg}\,\operatorname{max}}\;}

\newcommand{\Ran}{\mathbb{R}^n}
\newcommand{\p}{\mathcal{P}}
\newtheorem{defin}{Definizione}
\newtheorem{lem}{Lemma}
\newtheorem{teo}{Teorema}
\newtheorem{prop}{Proposizione}


\bibliography{tesi}

\begin{document}

%\input{FrontPage}

\tableofcontents %indice


\chapter{Introduzione}
\setlength{\epigraphwidth}{.95\textwidth}
\begin{epigraphs}
\qitem{Da quella notte\\
\`e entrato il gatto\\
con una testa in bocca\\
e me l'ha messa\\
tra i miei circuiti\\
ed ha ripreso vita\\
ed \`e da allora\\
che ho la coscienza\\
perch\`e di notte sogno\\
di quella notte\\
che \`e entrato il gatto\\
con una testa in bocca.}
{---\textsc{IL SOGNO DEL COMPUTER} --- Musica Per Bambini (Dei Nuovi Animali)}
\end{epigraphs}
\newpage
\label{ch:introduction}
\input{introduzione}
%\input{FormalismNotation}






\chapter{Conclusioni}
Qua ci stanno le conclusioni.

\end{document}