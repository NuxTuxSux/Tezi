\documentclass[a4paper,12pt]{report}
\usepackage{amsmath}
%[
	% Mi fa schifo \emptyset di \amssymb, evito che lo sovrascriva
	\let\oldemptyset\emptyset
	\usepackage{amssymb}
	% Decider\`o di essere nostalgico o mi aprir\`o al cambiamento?
%]

\usepackage{amsthm}
\usepackage{bm}
\usepackage{varioref}
%RISOLVERE
\usepackage{mathabx}

%\usepackage[italian]{babel}
%\usepackage[utf8]{inputenc}
\author{Nunzio Turtulici}
\title{Politopi Regolari Astratti}

\newcommand{\Rn}{\mathbb{R}^n}
\newcommand{\p}{\mathcal{P}}
\theoremstyle{plain}
\newtheorem{teo}{Teorema}[chapter]

\theoremstyle{definition}
\newtheorem{defin}[teo]{Definizione}
\newtheorem{lem}[teo]{Lemma}
\newtheorem{prop}[teo]{Proposizione}
\newtheorem{oss}[teo]{Osservazione}
\newtheorem{corol}[teo]{Corollario}
\newtheorem{example}[teo]{Esempio}

\setcounter{chapter}{1}

\newcommand\locallabel[1]{\label{\currentprefix_#1}}
\newcommand\localref[1]{\ref{\currentprefix_#1}}
\newcommand\point[1]{$\bm{(#1)}$\textbf{: }}
\newcommand\implication[2]{$\bm{(#1)\Rightarrow(#2)}$\textbf{: }}

\newcommand\flag[1]{\mathcal{F}(#1)}
\def\restrict#1{\raise-.5ex\hbox{\ensuremath|}_{#1}}


\begin{document}




\section{Politopi astratti}
In questo capitolo parleremo dei politopi astratti, un'astrazione assiomatica della struttura combinatoria dei politopi (concreti) di $\Rn$.
Essi incapsulano la
semplice struttura combinatoria delle facce di un politopo, essendo questa gi\`a molto ricca di informazioni circa il politopo stesso. Cominciamo con alcune 
definizioni.
\subsection{Definizioni}
Un insieme parzialmente ordinato $(\p,\leq)$ si dice \emph{politopo astratto di rango} $n$ (dove $rk\p:=n\in\mathbb{N})$ se soddisfa le
propriet\`a $(P1)-(P4)$ scritte 
di seguito. Per una migliore comprensione delle propriet\`a in questione introduciamo la terminologia di base. Gli elementi di $\p$ verranno
chiamati \emph{facce},
due facce $F,G\in\p$ verranno dette \emph{incidenti} se $F\leq G$ o $G\leq F$. Ricordiamo inoltre che una \emph{catena} di un insieme ordinato \`e un suo 
sottoinsieme totalmente ordinato. Nel caso dei politopi chiameremo \emph{lunghezza} di una catena $\Omega$ il numero $\left|\Omega\right|-1$ e, vista
l'importanza delle catene massimali, chiameremo quest'ultime \emph{bandiere}.
Denoteremo con $\mathcal{F}(\p)$ l'insieme di tali bandiere. Ovviamente ogni catena di $\p$ sar\`a contenuta in un elemento di $\mathcal{F}(\p)$.
Enunciamo i primi due assiomi di politopo astratto. Sia $\p$ un insieme ordinato.
\begin{itemize}
\item (P1) $\p$ contiene un minimo e un massimo, che chiamiamo $\bot_\p$ e $\top_\p$ (o - ove non creasse confusione $\bot$ e $\top$ o ancora
con un piccolo abuso di notazione $\emptyset$ e $\p$) rispettivamente.
\item (P2) Le bandiere di $\p$ hanno tutte la medesima lunghezza
\end{itemize}
Se un insieme ordinato $\p$ soddisfa le propriet\`a $(P1)-(P2)$ diremo \emph{rango} di $\p$ il numero intero  $\left|\Phi\right|-2$, dove
$\Phi$ \`e una sua bandiera. Per $(P2)$ la definizione \`e ben posta.\\
Per enunciare i restanti assiomi di politopo astratto abbiamo bisogno di definire le sue \emph{sezioni}.
\begin{defin}
Date $F,G\in\p$ chiameremo sezione di $\p$ relativa a $F$ e $G$ il suo sottoinsieme
\begin{equation*}
G/F=\left\{H\in\p\mid F\leq H\leq G\right\}
\end{equation*}
Diremo inoltre che una sezione \`e propria se questa \`e distinta da $\p$ e da $\left\{\bot\right\}$.
\`E bene mostrare che una sezione di un poset che soddisfa $(P1)-(P2)$ soddisfa essa stessa $(P1)-(P2)$. Mostreremo in effetti (nel seguito) che una sezione
 di un politopo astratto continuer\`a ad essere un politopo astratto. Dimostriamo per ora il seguente lemma che ci permetter\`a, fra le altre cose,
 di definire un rango tra le facce di un poset che soddisfi $(P1)-(P2)$ e quindi nel seguito di un politopo astratto.
\end{defin}
\begin{lem}
\label{P1P2Sections}
Sia $\p$ un insieme parzialmente ordinato che soddisfi le propriet\`a $(P1)-(P2)$ e siano $G,F\in\p$ tali che $G\leq F$. Allora la sezione non vuota
$F/G$ soddisfa anch'essa le propriet\`a $(P1)-(P2)$
\end{lem}
\begin{proof}
La propriet\`a $(P1)$ \`e banale, essendo per definizione di $F/G$, gli elementi $F$ e $G$ massimo e minimo rispettivamente.\\
Mostriamo la pi\`u interessante $(P2)$, ossia che le bandiere di $F/G$ abbiano la stessa lunghezza. Siano $\Phi,\Psi\in\mathcal{F}(F/G)$
completiamo $\Phi$ a $\bar{\Phi}\in\mathcal{F}(\p)$. Allora, posti
\begin{equation*}
\Phi_0=\bar{\Phi}\cap G/\bot\text{\quad e\quad }\Phi_1=\bar{\Phi}\cap\p/F
\end{equation*}
possiamo estendere $\Phi$ a bandiera di $\p$ in modo \emph{compatibile} con $\bar{\Phi}$. Ossia, posto
\begin{equation*}
\bar{\Psi}=\Phi_0\cup\Psi\cup\Phi_1
\end{equation*}
avremo che $\bar{\Psi}\in\mathcal{F}(\p)$. \`E chiaro che $\bar{\Psi}$ sia una catena, essendo $G$ contemporaneamente maggiorante per $\Phi_0$ e
minorante per $\Psi$ e $F$ maggiorante per $\Phi$ e minorante per $\Phi_1$.\\
Se inoltre, avessimo $H\in\p$ tale che $\bar{\Psi}\cup\left\{H\right\}>\bar{\Psi}$ continui ad essere una catena, allora si presenterebbero tre casi.\\
\textbf{Caso $\bm{H<G}$:}
In tal caso avremmo $H\in G/\bot$ pertanto, essendo $H\notin\Psi\supseteq\Phi_0$, avremmo che $H\notin\bar{\Phi}$, tuttavia
\begin{equation*}
\left(\Phi_0\cup\left\{H\right\}\right)\cup\Phi\cup\Phi_1
\end{equation*}
sarebbe una catena, contro la massimalit\`a di $\bar{\Phi}$.\\
\textbf{Caso $\bm{F<H}$:}
La dimostrazione di questo caso \`e simmetrica a quella del caso precedente.\\
\textbf{Caso $\bm{G\leq H\leq F}$:}
Se $H\in F/G$ allora, essendo $\bar{\Psi}\cup\left\{H\right\}$ una catena, lo sar\`a anche $\Psi\cup\left\{H\right\}$, ma
$H\notin\bar{\Psi}$ quindi $H\notin\Psi$ e $\Psi\cup\left\{H\right\}\supsetneq\Psi$, contro la massimalit\`a di $\Psi\in\mathcal{F}(F/G)$.
Abbiamo quindi mostrato che il completamento di $\Psi$ sopra definito risulta effettivamente una bandiera (di $\p$)
\begin{equation*}
\bar{\Psi}\in\mathcal{F}(\p)
\end{equation*}
A questo punto, contando gli elementi delle bandiere in questione (per l'ipotesi $(P2)$ su $\p$), abbiamo
\begin{gather*}
\left|\bar{\Psi}\right|=|\Phi_0|+|\Psi|+|\Phi_1|-2\\
|\bar{\Phi}|=|\Phi_0|+|\Phi|+|\Phi_1|-2
\end{gather*}
da cui $|\Phi|=|\Psi|$.
\end{proof}
\par
Se un poset $\p$ soddisfa le propriet\`a $(P1)-(P2)$, alla luce di \ref{P1P2Sections}, osserviamo che \`e possibile estendere la nozione di \emph{rango}
alle facce (elementi) di $\p$.\\
Infatti, per vedere le facce di un politopo come politopi a loro volta, ne considereremo le sezioni con la faccia minima, $\bot$,
cos\`i da identificare una faccia $F\in\p$ con l'insieme delle facce \emph{sotto} $F$ (sottoinsieme di $\p$ che risulta avere $F$ come massimo).
Se $F\in\p$ chiameremo quindi $rk_\p F:=rk(F/\bot)$ \emph{rango} della faccia $F$, ove non ci sia possibilit\`a di confusione scriveremo
pi\`u semplicemente $rk F:=rk_\p F$. Diremo inoltre che $F$ \`e una $i$-faccia.
Notiamo che $rk(\bot)=-1$ e $rk(F_n)=n$ queste facce sono ovviamente le uniche di $\p$ con tali ranghi e verranno dette
facce improprie di $\p$.
Definiamo inoltre
\begin{equation*}
\p_i=\left\{F\in\p\mid rkF=i\right\}
\end{equation*}
Chiameremo le facce di $\p$ di rango $0$ \emph{vertici} di $\p$, quelle di rango $1$ \emph{vertici} e quelle di rango $n-1$ \emph{faccette}. Data 
una qualunque faccia $F\in\p$ diremo che la sezione $P/F$ \`e la sua \emph{co-faccia}.\\
Mostreremo nell'immediato seguito che se $F,G\in\p$ con $F\leq G$ allora
\begin{equation*}
rk(G/F)=rkG-rkF-1
\end{equation*}
In particolare $rk(F/\bot)=rkF$ e $rk(P/F)=n-rkF-1$. Chiameremo quest'ultimo \emph{co-rango} di $F$.
Parlando di una catena $\Omega$ di $\p$ chiameremo \emph{tipo} di $\Omega$ l'insieme ordinato dei ranghi dei suoi elementi e scrivendo 
$\Omega=\left\{F_{i_1},\dots,F_{i_l}\right\}$ assumeremo implicitamente che $rkF_{i_t}=i_t$, che $i_1<\cdots<i_l$ e, quindi,
che $\Omega$ sia di tipo $(i_1,\dots,i_l)$.
Talvolta ometteremo nell'enumerazione di una catena le sue facce improprie.\\

Avendo definito il rango delle facce, sar\`a utile vedere come questo concetto di \emph{livello} nel poset permetta
di misurare in qualche modo l'ampiezza delle sezioni.
Proseguiamo quindi col seguente lemma che ci permetter\`a di collegare numericamente i ranghi delle sezioni, i ranghi delle facce \emph{nelle sezioni}
ai ranghi intrinseci delle facce.
\begin{lem}
\label{lem:SectionRanks}
Sia $\p$ un poset che soddisfi $(P1)-(P2)$, $G<F\in\p$ e inoltre $H\in F/G$, allora
\begin{itemize}
\item $rk(F/G)=rkF-rkG-1$
\item $rk_{F/G}H=rk_\p(H/G)(=rk_\p H-rk_\p G-1)$
\end{itemize}
\end{lem}
\begin{proof}
Per quanto riguarda il primo punto, osserviamo che sia $F=F/\bot$ che $G=G/\bot$ e $F/G$ soddisfano le propriet\`a $(P1)-(P2)$ per
\ref{P1P2Sections}, pertanto, scelti
\begin{gather*}
\underline{\Phi}\in\mathcal{F}(G)=\mathcal{F}(G/\bot),\\
\overline{\Phi}\in\mathcal{F}(F/G)
\end{gather*}
avremo che
\begin{equation*}
\Phi=\underline{\Phi}\cup\overline{\Phi}\in\mathcal{F}(F)=\mathcal{F}(F/\bot)
\end{equation*}
quindi, avremo
\begin{gather*}
rkG=|\underline{\Phi}|-2\\
rkF=|\overline{\Phi}|-2\\
rk(F/G)=|\Phi|-2
\end{gather*}
Ed essendo $\underline{\Phi}\cap\overline{\Phi}=\left\{G\right\}$, otterremo
\begin{gather*}
rkG+rk(F/G)=|\underline{\Phi}|-2+|\overline{\Phi}|-2=\\
|\Phi|+1-4=|\Phi|-2-1=rF-1
\end{gather*}
Da cui il primo punto. Per quanto riguarda il secondo punto ricordiamo che $rk_{F/G}(H)$ \`e per definizione la lunghezza di una bandiera
della sezione $H/G$ del poset $F/G$, minorata di due. Ma questa \`e precisamente la sezione $H/G$ vista come poset a s\'e stante, per cui
i due ranghi coincidono.
\end{proof}
A questo punto, per enunciare il terzo assioma di politopo, dobbiamo cominciare a parlare di \emph{connessione}.
\begin{defin}
Un insieme parzialmente ordinato $\p$ che gode delle propriet\`a (P1)-(P2) si dir\`a \emph{connesso} se
\begin{equation*}
\forall F,G\in\p\quad\exists F=:H_0,\cdots,H_l:=G\in\p\qquad \text{tali che }H_i\text{ e }H_{i+1}\text{ sono incidenti}
\end{equation*}
Considereremo per definizione $\p$ connesso nel caso $rk\p=1$.
\end{defin}
Arriviamo finalmente all'enunciazione del terzo assioma definendo la connessione forte di un insieme parzialmente ordinato.
\begin{defin}
Sia $\p$ un insieme parzialmente ordinato che gode delle propriet\`a (P1)-(P2). Diremo che $\p$ \`e \emph{fortemente connesso} se
\`e connessa ogni sua sezione.
\end{defin}
Come preannunciato il prossimo e terzo assioma di politopo astratto \`e
\begin{itemize}
\item (P3) $\p$ \`e fortemente connesso
\end{itemize}
Nel seguito utilizzeremo una variante di (P3) che fa uso delle bandiere del politopo. A tal fine diamo le seguenti
\begin{defin}
Sia $\p$ un insieme parzialmente ordinato che soddisfa (P1)-(P2). Due bandiere $\Phi,\Psi\in\mathcal{F}(\p)$ si dicono \emph{adiacenti}
se differiscono per un solo elemento. Se $\left\{F\right\}=\Phi\cap\Psi$ e $i=rkF$ diremo anche che $\Phi$ e $\Psi$ sono $i$-adiacenti.\\
\end{defin}
\begin{defin}
Sia $\p$ un insieme parzialmente ordinato che soddisfa (P1)-(P2). Allora diremo che $\p$ \`e connesso per bandiere se $\forall\Phi,\Psi\in\mathcal{F}(\p)$
esistono $\Phi_i\in\mathcal{F}(\p)$ per $i=0,\dots,k$ tali che
\begin{gather*}
\Phi_0=\Phi\\
\Phi_k=\Psi\\
\Phi_i\text{ e }\Phi_{i+1}\text{ sono adiacenti}
\end{gather*}
\end{defin}
In modo analogo a quanto visto per la connessione definiamo la connessione forte per bandiere.
\begin{defin}
Se $\p$ \`e un insieme parzialmente ordinato che soddisfa (P1)-(P2) allora diremo che $\p$ \`e \emph{fortemente connesso per bandiere} se
ogni sua sezione \`e connessa per bandiere
\end{defin}

Enunciamo senza dimostrare la seguente ovvia osservazione che sar\`a comoda nel teorema seguente.
\begin{oss}
Sia $\p$ un politopo, $\Phi\in\mathcal{F}(\p)$ una sua bandiera e $F,G\in\Phi$. Allora le facce $F$ e $G$ sono connesse. Inoltre se
$\Psi$ \`e una bandiera di $\p$ incidente a $\Phi$ e $H\in\Psi$ una sua faccia, abbiamo che $H$ \`e connessa a $F$ (e a $G$).
\end{oss}

\begin{prop}
\label{prop:P3Flags}
Sia $\p$ un insieme parzialmente ordinato che soddisfa (P1)-(P2). Allora $\p$ \`e fortemente connesso se e solo se \`e fortemente
connesso per bandiere.
\end{prop}
\begin{proof}
Procediamo per induzione su $n=rk\p$, non essendoci nulla da dimostrare per $n\leq 1$ supponiamo $n\geq 2$. Mostriamo che se $\p$ \`e
fortemente connesso allora esso \`e fortemente connesso per bandiere.\\
Siano $\Phi,\Psi\in\mathcal{F}(\p)$, due bandiere che per ora considereremo non disgiunte. Sia allora $\Omega=\Phi\cap\Psi=\left\{F_{j_1},\dots,F_{j_k}\right\}$ 
la loro parte comune. Posto $F_{j_0}=\bot$ (la faccia minima di $\p$) avremo che le sezioni $F_{j_i}/F_{j_{i-1}}$ sono fortemente connesse e, quindi,
per ipotesi induttiva (essendo queste di rango strettamente minore di $n$), fortemente connesse per bandiere. \`E quindi possibile \emph{muovere} la parte di
$\Phi$ contenuta in ognuna di queste sezioni nella relativa parte in $\Psi$. Per fissare le idee consideriamo, ad esempio, la sezione
$\mathcal{G}_0:=F_{j_1}/F_{j_0}$.\\
Essendo $\Phi\cap\mathcal{G}_0,\Psi\cap\mathcal{G}_0\in\mathcal{F}(\mathcal{G}_0)$ due bandiere della sezione $\mathcal{G}_0$ queste saranno connesse da una 
successione $\Upsilon_0,\dots,\Upsilon_t\in\mathcal{F}(\mathcal{G}_0)$. Pertanto le $\widetilde{\Upsilon}_l:=(\Phi\setminus\mathcal{G}_0)\cup\Upsilon_l$ per
$l=0,\dots,t$ costituiranno una successione che connette la bandiera $\Phi$ alla bandiera $\Phi\setminus\mathcal{G}_0\cup(\Psi\cap\mathcal{G}_0)$
costituita da elementi che coincidono su $\Omega$.
Connettendo - relativamente alla sezione $\mathcal{G}_0$ - le bandiere $\Phi$ e $\Psi$. Iterando il procedimento per ognuna delle restanti sezioni
$\mathcal{G}_1,\dots,\mathcal{G}_i$ otteniamo la tesi (nel caso $\Omega\ne\emptyset$).\\
Consideriamo adesso il caso $\Omega=\emptyset$, nel quale dovremo quindi semplicemente mostrare che le bandiere $\Phi$ e $\Psi$ sono connesse.
Siano a tal fine $G_0\in\Phi$ e $G_l\in\Psi$ due facce qualunque.\\
Queste saranno estremi di una successione $(G_i)_{i=0}^l$ di facce di $\p$
ognuna adiacente alla successiva (per la connessione di $\p$).\\
Essendo $G_i$ e $G_{i+1}$ adiacenti per ogni $i=0,\dots,l-1$, queste saranno contenute in una bandiera $\Phi_i$ (in quanto catena massimale). Notiamo che
$G_{i+1}\in\Phi_i\cap\Phi_{i+1}$, pertanto per il punto precedente ($\Omega\neq\emptyset$) le bandiere $\Phi_i$ e $\Phi_{i+1}$ sono connesse da una successione
in $\mathcal{F}(\p)$. Pertanto, unendo tali successioni per $i=0,\dots,l-1$ si ottiene una successione che connette $\Phi$ e $\Psi$.\\
Per quanto riguarda l'implicazione inversa supponiamo $\p$ fortemente connesso per bandiere, siano $F,G\in\p$ con $G<F$ mostriamo quindi che la sezione $F/G$
\`e connessa.\\
Siano quindi $H,H'\in F/G$, mostriamo che queste sono connesse. Siano $\Phi,\Phi'\in\mathcal{F}(F/G)$ tali che $H\in\Phi$ e $H'\in\Phi'$, allora 
se $\Phi=\Psi_0,\dots,\Psi_k=\Phi'$ \`e una successione di bandiere adiacenti in $F/G$, per l'osservazione [PRECEDENTE!] (e per la transitivit\`a
della relazione di connessione tra le facce) avremo che $H$ e $H'$ saranno connesse da una successione di facce adiacenti in $F/G$.
\end{proof}
Alla luce dell'ultima propriet\`a abbiamo che (sotto le ipotesi $(P1)-(P2)$) $(P3)$ \`e equivalente alla:
\begin{itemize}
\item (P3') $\p$ \`e \emph{fortemente connesso per bandiere}
\end{itemize}
L'ultima propriet\`a che definisce un politopo astratto fa in modo che questi finora non troppo particolari reticoli modellino quelli costituiti
dalle facce dei politopi concreti nello spazio euclideo (insiemi che risultano effettivamente reticoli se considerati con l'inclusione).
La cosiddetta \emph{diamond property} esprime la struttura delle sezioni unidimensionali del politopo garantendo, per esempio, che un vertice in un
poligono sia contenuto in esattamente due lati, che uno spigolo di un poliedro sia contenuto in esattamente due faccette o, sempre nel caso tridimensionale,
che un vertice contenuto in una faccetta si possa \emph{estendere} in esattamente due modi a spigolo rimanendo nella faccia in questione.\\
Enunciamo quindi il quarto ed ultimo assioma di politopo astratto.
\begin{itemize}
\item (P4) Per ogni $0\leq i\leq n-1$ se $F,G\in\p$ sono tali che $rkF=i-1$ e $rkG=i+1$, allora esistono esattamente due facce $H\in\p$ tali che
\begin{equation*}
F<H<G
\end{equation*}
\end{itemize}

Notiamo che $(P4)$ asserisce che ogni sezione unidimensionale \`e a forma di \emph{diamante} [INSERIRE DIAGRAMMA!], da cui il nome dell'assioma. Notiamo
inoltre che, essendo contemplato il caso $i=0$ abbiamo da questa propriet\`a che ogni $1$-faccia (lato, spigolo, \dots) ha esattamente due vertici 
(ricordando che la faccia minima $\bot$ \`e contenuta in ogni altra faccia del politopo).\\
In modo pi\`u compatto possiamo esprimere $(P4)$ come segue
\begin{oss}
Sia $\p$ politopo astratto e $F,G\in\p$ con $rkG = rkF + 2$ allora
\begin{equation*}
\left|G/F\right|=4
\end{equation*}
\end{oss}
Osserviamo come $(P4)$ sia un assioma di \emph{regolarit\`a} del reticolo e come abbia una ricaduta al livello delle bandiere del politopo,
permettendoci cio\`e di definire, data una bandiera, le sue

\begin{defin}[Bandiere $i$-adiacenti]
Sia $\Phi\in\mathcal{F}(\p)$ e $0<i<n$ allora esiste esattamente un'altra bandiera $\Psi\in\mathcal{F}(\p)$ che differisca da $\Phi$ per la sua faccia $i$-esima.
Tale $\Psi$ sar\`a quindi adiacente a $\Phi$ e, alla luce della sua unicit\`a, porremo per comodit\`a $\Phi^i=\Psi$
Reiterando il procedimento porremo ricorsivamente (per $k\geq 1$)
\begin{equation*}
\Phi^{i_1\cdots i_k i_{k+1}}=(\Phi^{(i_1\cdots i_k)})^{i_{k+1}}
\end{equation*}
\end{defin}
Per il successivo studio dei gruppi di simmetria di $\p$ cominciamo a mostrare delle fondamentali quanto elementari propriet\`a delle bandiere $i$-incidenti
ad una bandiera.

\begin{lem}
Siano $\Phi\in\mathcal{F}(P)$ e $0\leq i\leq k\leq n-1$, allora valgono le seguenti:
\begin{itemize}
\item $\Phi^{ii} = \Phi$
\item $\Phi^{ij} = \Phi^{ji}$\quad per $j\geq i+2$
\end{itemize}
\end{lem}

\begin{defin}
Un politopo astratto di rango $n$ \`e un poset che soddisfa le propriet\`a $(P1)-(P4)$. Indicheremo con $rk\p$ il rango di $\p$.
\end{defin}

In genere ometteremo l'aggettivo \emph{astratto} e, ove serva discriminare, indicheremo i politopi immersi in $\Rn$ come \emph{concreti} o
\emph{convessi}.\\
Come preannunciato, seguendo la nomenclatura dei politopi concreti chiameremo un politopo di rango $2$ \emph{poligono} e uno di rango $3$
\emph{poliedro}.
\begin{oss}
\`E bene osservare che dagli assiomi di politopo nulla ci garantisce che questo sia finito. Infatti esistono politopi infiniti, chiameremo un tale
politopo (di rango $n$) \emph{$n$-apeirotopo}
Inoltre, seguendo il caso generale, un $2$-apeirotopo sar\`a detto \emph{apeirogono} e un $3$-apeirotopo \emph{apeiroedro}
\end{oss}
[AGGIUNGERE DIAGRAMMI!]
Procediamo con degli esempi e studiamo i casi pi\`u banali, ossia i politopi astratti di rango al pi\`u $2$.\\
L'unico politopo di rango $-1$ \`e il politopo con un'unica faccia $\bot(=\top)$, questo sar\`a il politopo \emph{vuoto}.
Passando al rango $0$, ossia i politopi \emph{punti}, un politopo $\p$ avr\`a un diagramma costituito
da due nodi e un arco che li connette, ossia $\p=\left\{\bot,F_0\right\}$ con ovviamente $\bot<F_0$. $F_0$ sar\`a il punto stesso.\\
Nel caso in cui $rk\p=1$, invece, avremo un diagramma a diamante, essendo $rkF_1-rk\bot=2$, cio\`e $\p$ coincide con la sua 
$1$-sezione $F_1/\bot$. Quindi $\p$ sar\`a un segmento e le sue due facce proprie saranno i suoi due vertici.\\
Il caso $rk\p=2$ \`e quello dei poligoni [DIAGRAMMI!].	\\
Notiamo che un politopo potrebbe analogamente essere definito ricorsivamente. Ossia
\begin{oss}
Un politopo di rango $n\leq 1$ \`e un poset di cui al diagramma sopra [DIAGRAMMA!], invece per $n\geq 2$ un politopo di rango $n$ \`e un poset $\p$ che
soddisfi $(P1)$ e tale che (ricorsivamente) ogni sua ($n-1$)-sezione sia un ($n-1$)-politopo.
\end{oss}
Possiamo interpretare questa definizione alternativa immaginando di \emph{incollare} dei $(n-1)$-politopi lungo le loro faccette, ad ottenere un
$n$-politopo che contenga questi come faccette, ossia che li abbia come particolari $(n-1)$-sezioni.\\
Vediamo adesso - come promesso - che le sezioni di un politopo astratto risultano essere, come poset, anch'essi politopi.
\begin{teo}
\label{teo:SectionIsPoly}
Sia $\p$ un politopo di rango $n$ e $G<F\in\p$ due sue facce. Allora $F/G$ risulta essere un politopo di rango $rk(F/G)=rkF-rkG-1$
\end{teo}
\begin{proof}
Alla luce di \ref{P1P2Sections}, per mostrare che $F/G$ \`e un politopo dobbiamo mostrare che esso soddisfa $(P3)-(P4)$.\\
$\bm{(P3)}$\textbf{:} Basta notare che una qualunque sezione di $F/G$ \`e anche sezione di $\p$, risulta pertanto connessa.\\
$\bm{(P4)}$\textbf{:} Per vedere che vale la \emph{diamond property} anche nella sezione $F/G$, supponiamo che $rk(F/G)\geq1$.
Siano $H<K\in F/G$ con $rk_{F/G}H=j-1$ e $rk_{F/G}K=j+1$ con $0\leq j\leq rk(F/G)-1$, allora per \ref{lem:SectionRanks}, posto
$i=j+rk_\p G+1$ si ha
\begin{gather*}
rk_\p K=rk_{F/G}K+rk_\p G+1=j+2+rk_\p G=i+1\text{ e}\\
rk_\p H=rk_{F/G}H+rk_\p G+1=j+rk_\p G=i-1
\end{gather*}
pertanto, applicando $(P4)$ relativamente a $\p$ avremo che la sezione $K/H$ ha precisamente due facce proprie, concludiamo notando che tale
sezione risulta essere la stessa in $\p$ e in $F/G$
\end{proof}
Alla luce di quanto visto, essendo le sezioni di $\p$ esse stesse politopi, viene naturale considerare le facce di $\p$ (gli elementi del poset)
come sottostrutture, identificando ogni $F\in\p$ con la sezione $F/\bot$.
\begin{corol}
Sia $\p$ un $n$-politopo e $F\in\p_k$ una sua $k$-faccia, allora $F$ \`e un $k$-politopo.
\end{corol}
\begin{proof}
Per \ref{teo:SectionIsPoly} basta notare che $rk(F/\bot)=rkF-rk\bot-1=rkF$
\end{proof}

Proseguiamo lo studio di queste strutture definendo i loro morfismi e le loro simmetrie.

\subsection{Morfismi e Gruppi di simmetria}
Definiamo adesso i morfismi tra politopi.
\begin{defin}[Morfismo di Politopi]
Siano $\p,\mathcal{Q}$ due politopi (di rango qualunque). Una funzione
\begin{equation*}
\phi:\p\longrightarrow\mathcal{Q}
\end{equation*}
\`e detta \emph{morfismo di politopi} se essa \`e crescente, ossia se $\forall F,G\in\p\quad F\leq G$ implica $\phi F\leq\phi G$.
\end{defin}
Notiamo che la definizione ammette morfismi tra politopi di rango diverso. Inoltre diremo che un morfismo $\phi$ \`e un isomorfismo se questo \`e
biettivo e inoltre $\phi^{-1}$ continua ad essere un morfismo (condizione necessaria, come nell'analogo caso topologico).
Se tra due politopi $\p$ e $\mathcal{Q}$ esiste un isomorfismo, diremo che questi sono isomorfi e scriveremo $\p\cong\mathcal{Q}$.\\
Un ruolo importante nella struttura di un politopo viene giocato dalle sue simmetrie che, come normalmente accade in altre strutture, costituiscono un
gruppo.\\
\begin{defin}
Sia $\p$ un politopo. Chiameremo un isomorfismo $\phi:\p\longrightarrow\p$ \emph{automorfismo} di $\p$. Denoteremo con $\Gamma(\p)$ l'insieme
di tali automorfismi. Quest'ultimo risulter\`a chiaramente essere un gruppo, detto \emph{gruppo di simmetria} di $\p$.
\end{defin}
[QUI CI STAREBBE BENE LA CLASSIFICAZIONE DEI POLIGONI PI\`U APEIROGON!]

Definiamo un altro importantissimo concetto nei politopi. Come si pu\`o notare, dalla definizione (necessariamente) simmetrica di politopo
tra ``grande'' e ``piccolo'', emerge il concetto di \emph{dualit\`a}, che definiamo come segue.
\begin{defin}
Sia $\phi:\longrightarrow\mathcal{Q}$ \`e una biiezione tra due politopi che inverte l'ordine diremo che $\phi$ \`e una \emph{dualit\`a} tra
$\p$ e $\mathcal{Q}$ e diremo che $\p$ e $\mathcal{Q}$ sono \emph{duali}
\end{defin}

\begin{oss}
\label{oss:DualUnicity}
Osserviamo che se $\p$, $\mathcal{Q}$ e $\mathcal{Q}'$ sono politopi e $\phi:\p\longrightarrow\mathcal{Q}$ e $\phi':\p\longrightarrow\mathcal{Q'}$
sono antiisomorfismi di poset, allora $\mathcal{Q}\cong\mathcal{Q}'$, da cui l'essenziale unicit\`a del politopo duale.
\end{oss}
\begin{proof}
Per ottenere l'asserto \`e sufficiente notare che $\phi'\phi^{-1}:\mathcal{Q}\longrightarrow\mathcal{Q}'$ \`e una biiezione che preserva l'ordine
\end{proof}

\begin{defin}
\label{def:DualPolytope}
Sia $\p=(\p,\leq)$ un politopo, indichiamo con $\widecheck{\p}=(\p,\geq)$ il suo poset opposto e se $F\in\p$, quando considereremo quest'ultima
come elemento di $\widecheck{\p}$, la denoteremo con $F^*$
\end{defin}

\def\currentprefix{prop:DualPolytope}
\begin{prop}[Politopo duale]
\label{prop:DualPolytope}
Sia $\p$ un $n$-politopo, allora il poset definito in \ref{def:DualPolytope}, $\widecheck{\p}$, \`e un $n$-politopo. Inoltre $\widecheck{\p}$ \`e duale a
$\p$. Alla luce di \ref{oss:DualUnicity} chiameremo $\widecheck{\p}$ \emph{il} duale di $\p$. Chiaramente si ha $\check{\widecheck{\p}}=\p$.
Inoltre abbiamo che
\begin{enumerate}
\item\locallabel{1} Se $G<F\in\p$, allora $\widecheck{F/G}=G^*/F^*$
\item\locallabel{2} Se $F\in\p$ allora $rkF^*=n-rkF-1$
\end{enumerate}
\end{prop}
\newcommand\localprefix{tuasorella}
\begin{proof}
Per ottenere la tesi, una volta mostrato esplicitamente che $\widecheck{\p}$ \`e un politopo \`e sufficiente notare che
$id_\p:\p\longrightarrow\widecheck{\p}$ \`e una biiezione che inverte l'ordine, risultando quindi una dualit\`a.\\
Pertanto mostriamo che $\widecheck{\p}$ soddisfa $(P1)-(P4)$.\\
$\bm{(P1)}$\textbf{:} Chiaramente il massimo di $\p$ risulter\`a essere minimo di $\widecheck{\p}$ e viceversa.\\
$\bm{(P2)}$\textbf{:} Basta notare che $\mathcal{F}(\widecheck{\p})=\mathcal{F}(\p)$, essendo ogni bandiera semplicemente ordinata
nel verso opposto.\\
$\bm{(P3)}$\textbf{:} Notiamo che se $F>G$ sono facce di $\p$, allora in $\widecheck{\p}$ avremo che, insiemisticamente, $F/G=G^*/{F^*}$ (in effetti
come poset si ha
$\widecheck{F/G}=G^*/{F^*}$, inoltre se due facce $H,K\in F/G$ sono incidenti, lo saranno anche $H^*$ e $K^*$ in $G^*/{F^*}$ e, pertanto, 
si preserver\`a la pi\`u generale relazione di connessione anche in $G^*/{F^*}$. Concludiamo che essendo la sezione $F/G$ di $\p$ connessa,
sar\`a connessa anche la sezione $G^*/{F^*}$ di $\widecheck{\p}$.\\
$\bm{(P4)}$\textbf{:} Notiamo che, se $G<F\in\p$, essendo $rk_\p(F/G)=rk_{\widecheck{\p}}(G^*/{F^*})$ le sezioni di rango $1$ in $\p$ saranno,
invertito l'ordine, sezioni di rango 1 anche in $\widecheck{\p}$, inoltre essendo queste uguali insiemisticamente, avremo che $(P4)$ sar\`a soddisfatto 
anche da $\widecheck{\p}$.
Per mostrare che $rk\p=rk\widecheck{\p}$ \`e sufficiente notare che $\mathcal{F}(\widehat{\p})=\mathcal{F}(\p)$.\\
Una volta visto che $\widecheck{\p}$ \`e un $n$-politopo passiamo a mostrare \localref{1}.\\
Per dimostrare che questa risulta un'uguaglianza \emph{come poset} prendiamo $H_1,H_2\in\widecheck{F/G}$ e supponiamo che sia
$H_1\leq_{\widecheck{F/G}} H_2$, allora
\begin{equation*}
G\leq H_2\leq H_1\leq F
\end{equation*}
quindi
\begin{equation*}
G^*\geq H_2^*\geq H_1^*\geq F^*
\end{equation*}
Pertanto, notando che come elementi $H_1=H_1^*$ e $H_2=H_2^*$ avremo che $H_1,H_2\in G^*/F*$ e inoltre $H_1\leq_{G^*/F^*}H_2$.\\
Dimostriamo quindi che vale \localref{2}. Sia $F\in\p$ allora, per la definizione di rango di una faccia, per \localref{1}, \ref{lem:SectionRanks} e
per la prima parte della tesi, si ha
\begin{gather*}
rkF^*=r(F^*/\p^*)=rk(\widecheck{\p/F})=r(\p/F)=rk\p-rkF-1=n-rkF-1
\end{gather*}

\end{proof}

Proseguiamo mostrando un altro importante politopo duale a $\p$, che d\`a quindi un'alternativa definizione naturale di politopo duale.

\begin{oss}
Sia $\p$ un $n$-politopo, allora il poset $\mathcal{Q}=\left\{\p/F|F\in\p\right\}$ col naturale ordine dato dall'inclusione risulta essere un
$n$-politopo duale a $\p$
\end{oss}
\begin{proof}
Mostriamo che il poset $\mathcal{Q}$ risulta essere antiisomorfo a $\p$ pertanto, per \ref{oss:DualUnicity}, isomorfo come poset a $\widecheck{\p}$.
Essendo, per \ref{prop:DualPolytope}, $\widecheck{\p}$ un politopo avremo la tesi.
Definiamo quindi
\begin{gather*}
\phi:\p\longrightarrow\mathcal{Q}\\
\qquad F\longmapsto\p/F
\end{gather*}
Chiaramente $\phi$ risulta suriettiva per la definizione di $\mathcal{Q}$. Notiamo che \`e anche iniettiva avendo che per ogni $F\in\p$ risulta
$F=\min\p/F$ ($\min$ \`e inversa sinistra di $\phi$). Mostriamo che $\phi$ inverte l'ordine. Siano $F,G\in\p$, con $F\leq G$, allora se
$H\in\p/G=\phi(G)$ abbiamo $H\geq G$, da cui $H\geq F$, pertanto $H\in\p/F=\phi(F)$, pertanto $F\leq G$ implica che $\phi(F)\supseteq\phi(G)$.
\end{proof}

\begin{oss}[Rango morfismi-dualit\`a]
\label{morphRank}
\`E bene notare che se $\phi:\p\longrightarrow\mathcal{Q}$ \`e un isomorfismo tra due politopi $\p$ e $\mathcal{Q}$ di rango $n$, allora esso
preserva il rango delle facce (quindi per ogni $i=-1,\dots,n$ si ha che $\phi\p_i=\mathcal{Q}_i$) e se invece tale $\phi$ \`e una dualit\`a tra
$\p$ e $\mathcal{Q}$ allora questa manda $i$-facce in co-$i$-facce. Ossia $\phi\p_i=\mathcal{Q}_{n-1-i}$.
\end{oss}
Proseguiamo con un lemma, che sar\`a utile nel seguito, che esprime la compatibilit\`a degli isomorfismi e antiisomorfismi con la relazione di
adiacenza delle bandiere.
\begin{lem}
\label{morphAdj}
Siano $\p$ e $\mathcal{Q}$ due $n$-politopi e $\phi:\p\longrightarrow\mathcal{Q}$ allora per ogni $i=0,\dots,n-1$
\begin{itemize}
\item Se $\phi$ \`e un isomorfismo allora $\phi(\Phi^i)=\phi(\Phi)^i$
\item se $\phi$ \`e una dualit\`a allora $\phi(\Phi^i)=\phi(\Phi)^{n-1-i}$
\end{itemize}
in entrambi i casi tale biiezione preserva l'adiacenza
\end{lem}
[EVENTUALE DIMOSTRAZIONE!]




\subsection{Azione di $\Gamma(\p)$}
Cominciamo adesso lo studio delle simmetrie di un politopo. Molto utile per tale studio \`e l'azione del gruppo di simmetria di un politopo su
suoi sottoinsiemi o su insiemi di suoi sottoinsiemi.\\
Per esempio per quanto osservato in \ref{morphRank}, avremo che se $\Gamma(\p)$ \`e il gruppo di simmetria di un politopo $\p$, allora questo
agisce su $\p_i$ per ogni $i=-1,\dots,n$. Il problema di questa rappresentazione di $\Gamma(\p)$ \`e che pu`o non essere fedele. Tuttavia
consideriamo la rappresentazione di $\Gamma(\p)$ come gruppo di permutazioni di $\mathcal{F}(\p)$.
\begin{oss}
Se $\p$ \`e un politopo e $\phi\in\Gamma(\p)$ un suo automorfismo, allora se $\Phi\in\mathcal{F}(\p)$ avremo che $\phi(\Phi)\in\mathcal{F}(\p)$.
Pertanto $\Gamma(\p)$ agisce su $\mathcal{F}(\p)$. Tale rappresentazione sar\`a molto utile per lo studio del gruppo di simmetria di $\p$.
\end{oss}

Quanto visto ci permette di mostrare una importante propriet\`a dell'azione del gruppo di simmetrie di un politopo sulle sue bandiere, il fatto
che questa sia \emph{libera}.[DEFINIRE AZIONE LIBERA?!]
\begin{lem}
\label{freeAction}
L'azione di $\Gamma(\p)$ su $\mathcal{F}(\p)$ \`e \emph{libera}.
\end{lem}
\begin{proof}
Sia $\sigma\in\Gamma(\p)$ e supponiamo che tale $\sigma$ fissi una bandiera $\Phi\in\mathcal{F}(\p)$, vogliamo mostrare che allora necessariamente
si avr\`a $\sigma=1$.\\
Per il lemma \ref{morphAdj}, essendo $\sigma(\Phi)=\Phi$, avremo che (per $i=0,\dots,n-1$)
\begin{equation*}
\sigma(\Phi^i)=\sigma(\Phi)^i=\Phi^i
\end{equation*}
Pertanto $\sigma$ fissa ogni bandiera adiacente a $\Phi$. Per la genericit\`a di $\Phi$ abbiamo mostrato che se un elemento di $\Gamma(\p)$ fissa una
bandiera, fissa tutte quelle a lei adiacenti. Possiamo concludere grazie all'equivalenza tra forte connessione e forte connessione per bandiere
(proposizione \ref{prop:P3Flags}) avendo mostrato che $\sigma(\Psi)=\Psi$ per ogni $\Psi\in\mathcal{F}(\p)$.
\end{proof}
Continuiamo con un utile criterio numerico che collega le cardinalit\`a del gruppo degli automorfismi di un politopo con quella dell'insieme delle sue
bandiere.
\begin{prop}
\label{regularAction}
Sia $\p$ un politopo \emph{finito}. Allora $\left|\Gamma(\p)\right|$ divide $\left|\mathcal{F}(\p)\right|$. Inoltre, nel caso in cui valga l'uguaglianza
$\left|\Gamma(\p)\right|=\left|\mathcal{F}(\p)\right|$ l'azione di cui al lemma \ref{freeAction} \`e transitiva oltrech\'e libera, quindi 
\`e un'azione \emph{regolare}.
\end{prop}
\begin{proof}
Posto $G=\Gamma(\p)$, consideriamo l'azione di $G$ su $\mathcal{F}(\p)$. Questa \`e libera per \ref{freeAction}, cio\`e per ogni bandiera
$\Phi\in\mathcal{F}(\p)$ si ha che $G_\Phi=\left\{1\right\}$. Pertanto, per l'equazione degli stabilizzatori e delle orbite abbiamo che
\begin{equation*}
\left|G\right|=[G:G_\Phi]=\left|\Phi^G\right|
\end{equation*}
per ogni bandiera $\Phi$.\\
A questo punto, per la decomposizione in orbite:
\begin{equation*}
\mathcal{F}(\p)=\bigsqcup_{i=1}^k \Phi_i^G
\end{equation*}
si ha quindi che $\left|\mathcal{F}(\p)\right|=k\left|G\right|$. Inoltre se vale l'uguaglianza tra $\left|\mathcal{F}(\p)\right|$ e $\left|G\right|$
avremo ovviamente $k=1$ cio\`e l'insieme delle bandiere di $\p$ risulter\`a un'unica orbita per l'azione di $G$.
\end{proof}
Essendo molto importanti per lo studio delle simmetrie dei politopi, nel seguito indicheremo, se $F\in\mathcal{F}(\p)$ e
$\Omega\subseteq\p$, con lieve abuso di notazione gli stabilizzatori di $F$ e $\Omega$ rispettivamente con $\Gamma(\p,F)=\Gamma(\p)_F$ e
$\Gamma(\p,\Omega)=\Gamma(\p)_\Omega$ facendo riferimento alle due diverse azioni del gruppo su (le facce di) $\p$ o su loro sottoinsiemi
(in particolare sul sotto $\Gamma(\p)$-insieme $\mathcal{F}(\p)\subset\mathcal{P}$ delle parti di $\p$.

Vediamo adesso un pi\`u ``concreto'' significato degli stabilizzatori di $\Gamma(\p)$. Infatti le simmetrie di $\p$ che fissano un suo particolare tipo di
catene risultano costituire un sottogruppo delle simmetrie di una sezione di $\p$ stesso.

\begin{prop}
\label{prop:SectionGroups}
Sia $\p$ un $n$-politopo. Allora
\begin{enumerate}
\item Se $\Omega=\left\{F\bot,F_0,\dots,F_i,F_j,\dots F_n\right\}$ \`e una \emph{catena} di $\p$, dove $i\leq j$ allora
\begin{equation*}
\Gamma(\p)_\Omega\hookrightarrow\Gamma(F_j/F_i)
\end{equation*}
\item Lo stabilizzatore di ogni vertice si immerge nel gruppo delle simmetrie della corrispondente figura al vertice [!DEFINIRE?]. Ossia, se $V\in\p_0$ \`e un vertice allora
\begin{equation*}
\Gamma(\p)_V\hookrightarrow\Gamma(\p/V)
\end{equation*}
\item Se $F\in\p_{n-1}$ \`e una faccetta, allora $\Gamma(\p)_F$ si immerge in $\Gamma(F)(=\Gamma(F/\bot))$
\end{enumerate}
\begin{proof}
Mostriamo $(1)$. L'immersione in questione \`e data da $\phi\mapsto\phi_{\mid F_j}$ per ogni $\phi\in\Gamma(\p)_\Omega$. Chiaramente - per la monotonia degli
automorfismi di $\p$ - la sezione $F_j/F_i$ \`e invariante per l'azione di $\Gamma(\p)_\Omega$,
inoltre tale azione \`e fedele. Infatti, se $\phi\in\Gamma(\p)_\Omega$ fissa una bandiera $\Phi'=\left\{F_i,G_{i+1},\dots,G_{j-1},F_j\right\}$ della sezione
$F_j/F_i$, allora osserviamo che $\Omega\cup\Phi'\in\mathcal{F}(\p)$ viene fissata da $\phi$ e per \ref{freeAction} questo implica $\phi=1$. Pertanto l'omomorfismo restrizione $\phi\mapsto\phi_{\mid F_j}$ risulta iniettivo.\\
$(2)$ e $(3)$ sono i due casi particolari estremi di $(1)$.
\end{proof}
\end{prop}

Definiamo lo \emph{scheletro} di un politopo.
\begin{defin}
Sia $\p$ un $n$-politopo e $0\leq k\leq n-1$, allora diremo $k$-scheletro di $\p$ il suo \emph{taglio a livello $k$} ossia
\begin{equation*}
skel_k(\p)=\left\{F\in\p\mid rkF\leq k\right\}
\end{equation*}
Chiameremo inoltre $skel_1(\p)$ \emph{grafo degli spigoli} di $\p$
\end{defin}

\begin{oss}
Notiamo che se $\p$ \`e un $n$-politopo e $0\leq k\leq n-1$, aggiungendo due facce improprie (come minimo e massimo) al poset $skel_k(\p)$ quest'ultimo
soddisfer\`a gli assiomi $(P1)-(P3)$ di $(k+1)$-politopo.
\end{oss}

Enunciamo a questo punto un interessante teorema circa i possibili percorsi sugli spigoli di un politopo, che ci far\`a apprezzare meglio
la visualizzazione del grafo degli spigoli di un politopo (di cui il $k$-scheletro \`e un'approssimazione).
Tale teorema risulta interessante e di facile visualizzazione nel caso di politopi concreti.
\begin{teo}
Sia $\p$ un $n$-politopo che risulti reticolo. Allora il suo grafo dei vertici, $skel_1(\p)$ \`e $n$-connesso, ossia ogni coppia di vertici
 \`e connessa in $skel_1(\p)$ da $n$ percorsi \emph{disgiunti}.
\end{teo}
\begin{proof}
[!FARE]
\end{proof}


\subsection{Simboli di Schl\"afli}
Proseguiamo lo studio definendo degli importantissimi invarianti numerici per i politopi, i cosiddetti \emph{numeri di Schl\"afli}. Questi danno un'idea
delle \emph{ampiezze} di alcune sezioni del politopo.
Notiamo che $(P4)$ (insieme chiaramente agli altri assiomi di politopo) forza la struttura delle $2$-sezioni di un politopo $\p$ ad avere oltre al solito
 massimo e minimo, precisamente due facce proprie.\\
Il caso pi\`u semplice \`e dato dal segmento, tali facce sono costituite dai due vertici del segmento. Un altro esempio \`e dato, nel caso dei $2$-politopi
ossia dei poligoni, dalle sezioni date da $\p/V$ se $V\in\p_0$ \`e un vertice. La propriet\`a $(P4)$ ci dir\`a in questo caso che ogni vertice di un
 poligono \`e contenuto (\`e luogo comune) di due lati. Questo esempio \`e molto facilmente estendibile per farci notare che in un politopo le $(n-2)$-facce
fanno da luogo d'``incollamento'' tra le faccette.\\
\begin{defin}
Sia $\p$ un $n$-politopo ($n\geq 2$), $1\leq i\leq n-1$ e $F\in\p_{i-2}, G\in\p_{i+1}$ con $G>F$. Allora poniamo
\begin{equation*}
p_i(F,G):=\left|(G/F)_i\right|
\end{equation*}
Ossia il numero delle $i$-facce della $2$-sezione relativa a $F$ e $G$. Osserviamo che \`e contemplato il caso in cui $p_i(F,G)=\infty$
\end{defin}
Caso particolarmente importante \`e quello in cui tali numeri dipendano semplicemente da $i$ e non dalle facce in questione, ossia in cui la ``forma''
locale del poset $\p$ dipenda solamente dall'\emph{altezza} a cui facciamo la sezione e non da quale sezione di quel tipo scegliamo. Tale situazione
\`e chiaramente un esempio di una sorta di \emph{regolarit\`a}. Procediamo con la definizione.
\begin{defin}[Equivelarit\`a]
Sia $\p$ un $n$-politopo e $1\leq i\leq n-1$. Se $\forall F,F'\in\p_{i-2}$ e $G,G'\in\p_{i+1}$
\begin{equation*}
p_i(F,G)=p_i(F',G')
\end{equation*}
Ossia, se i numeri $p_i(F,G)=:p_i$ dipendono solo dal ``livello'' $i$ della sezione e non dalle facce della stessa, diremo che $\p$ \`e \emph{equivelare}
con simbolo di Schl\"afli $\left\{p_1,\dots,p_{n-1}\right\}$.\\
Notiamo che nel caso del $1$-politopo definiamo suo simbolo di Schl\"afli la scrittura $\left\{\right\}$.
\end{defin} 
\begin{oss}
Notiamo che per i politopi pi\`u semplici per cui si possano fare delle $2$-sezioni, e cio\`e per il poligoni, l'equivelarit\`a \`e garantita. Inoltre
il simbolo di Schl\"afli $\left\{p_1\right\}$ caratterizza univocamente il poligono, che sar\`a appunto detto $p_1$-agono.
\end{oss}
Vediamo cosa succede alle sezioni di un politopo equivelare
\begin{oss}
Sia $\p$ un $n$-politopo equivelare con simbolo di Schl\"afli $\left\{p_1,\dots,p_{n-1}\right\}$.
Allora notiamo che ogni sua sezione di rango almeno $1$ \`e equivelare anch'essa. Inoltre se $F\in\p_{n-1}$ e $V\in\p_0$ sono rispettivamente una faccetta
e un vertice di $\p$ allora [? Le sezioni sono politopi, dimostrato?]
\begin{itemize}
\item $F$ \`e equivelare ed ha simbolo di Schl\"afli $\left\{p_1,\dots,p_{n-2}\right\}$
\item $\p/V$ \`e equivelare ed ha simbolo di Schl\"afli $\left\{p_2,\dots,p_{n-1}\right\}$
\end{itemize}
E, pi\`u in generale, se $F\in\p_{i-2}$ e $G\in\p_{j+1}$ con $i\leq j$ allora
\begin{itemize}
\item La sezione $G/F$ \`e equivelare con simbolo di Schl\"afli $\left\{p_i,\dots,p_j\right\}$
\end{itemize}
\end{oss}
Inoltre, per comodit\`a, nel caso in cui un politopo abbia simbolo (di Schl\"afli) con numeri consecutivi uguali scriveremo tali numeri con una
notazione \emph{esponenziale}, ossia, se l'$n$-politopo $\p$ ha simbolo $\left\{p_1,\dots,p_n\right\}$ con $p_i=\cdots=p_{i+k-1}$ allora scriveremo
tale simbolo pi\`u semplicemente come $\left\{p_1,\dots,p_{i-1},p_i^k,p_{i+k},\dots,p_{n-1}\right\}$.
Vediamo adesso un'interessante invariante dei politopi seguita da altri invarianti che potremo definire nel caso - per esempio - dei politopi regolari.
\begin{defin}
Sia $\p$ un $n$-politopo. Allora chiameremo
\begin{equation*}
f(\p):=\left(\left|\p_0\right|,\dots,\left|\p_{n-1}\right|\right)=:\left(f_0,\dots,f_{n-1}\right)
\end{equation*}
\emph{vettore delle facce} di $\p$
\end{defin}
Inoltre, nel caso in cui il politopo $\p$ sia \emph{sufficientemente} regolare in quanto a locali propriet\`a d'incidenza, ossia qualora le sue sezioni
risultino isomorfe a meno dei ranghi, sar\`a utile considerare la cardinalit\`a dell'insieme delle bandiere di tali sezioni.
\begin{defin}
Se $\p$ \`e un $n$-politopo tale che per ogni $-1\leq i\leq j\leq n$ le sezioni del tipo $G/F$ con $rkG=j$ e $rkF=i$ risultino fra loro isomorfe.
Allora definiamo
\begin{equation}
k_{ij}=k_{ij}(\p)=\left|\mathcal{F}(G/F)\right|\text{ se }F\in\p_i\text{ e }G\in\p_j\text{ con }-1\leq i\leq j\leq n
\end{equation}
\end{defin}
Per esempio, per ogni $n$-politopo avremo che $k_{-1n}=\left|\mathcal{F}(\p)\right|$ e - per la diamond property - anche che
$k_{i-1,i+1}=2$ per ogni $0\leq i\leq n-1$.
\begin{oss}
Inoltre se $n\geq 2$, per l'uniformit\`a delle sezioni, \emph{tagliando} a vari livelli, possiamo osservare che vale la seguente relazione
\begin{equation}
\label{kijf}
k_{-1,n}=f_0k_{0n}=f_{n-1}k_{-1,n-1}=k_{-1,i}f_ik_{in}
\end{equation}
\end{oss}
Continuiamo definendo un altro invariante numerico che esprime quanto sia \emph{sfaccettato} il politopo.
\begin{defin}
Sia $\p$ un $n$-politopo con le sezioni isomorfe a meno del rango, come sopra. Allora, per ogni $0\leq i,j\leq n$ definiremo $N_{ij}$ come
il numero di $j$-facce di $\p$ incidenti con una $i$-faccia fissata.
Notiamo che per l'ipotesi di regolarit\`a sulle sezioni tale definizione \`e ben posta.
\end{defin}
\begin{oss}
Siano $F\in\p_i$ e - come solito - $F_n,F\bot\in\p$ le facce improprie, allora
\begin{itemize}
\item $N_{ij}=f_{j-i-1}(F_n/F)$ se $i\leq j$
\item $N_{ij}=f_j(F/\bot)=f_j(F)$ se $i\geq j$
\end{itemize}
Notiamo che l'ultima uguaglianza deriva dal poter considerare la $i$-faccia come politopo a s\'e stante.
\end{oss}
Concludiamo, alla luce di questi invarianti numerici, con una formula (e la sua versione duale) per il conteggio per il numero di facce di un $n$-politopo.
Sostanzialmente possiamo contare le bandiere di $\p$ costruendole faccia per faccia, ossia conteggiando il numero di vertici e per ognuno quanti spigoli
ha sopra e cos\`i via fino alle faccette di $\p$. Per la regolarit\`a di $\p$ otterremo un semplice prodotto.
La versione duale del conteggio \`e svolto a partire dalla faccette, scendendo alle $(n-2)$-facce sotto di esse e cos\`i via fino ai vertici,
costruendo ogni bandiera in questione \emph{dall'alto verso il basso}.
\begin{oss}
Sia $\p$ un $n$-politopo ($n\geq 2$) che abbia tutte le sezioni tra facce di ranghi $i$ e $j$ fra loro isomorfe.
Allora essendo $k_{ij}$ e $N_{ij}$ come sopra, per il numero delle bandiere di $\p$ avremo il conteggio
\begin{equation}
\left|\mathcal{F}(\p)\right|=k_{-1n}=f_0N_{01}N_{12}\cdots N_{n-2,n-1}=f_{n-1}N_{n-1,n-2}N_{n-2,n-3}\cdots N_{1,0}
\end{equation}
\end{oss}

\section{Politopi Regolari}
Dopo aver introdotto alcune nozioni fondamentali per lo studio dei politopi astratti possiamo intrudurre il concetto di \emph{politopo regolare} e
cominciare a studiare tali oggetti. Innanzitutto \`e bene notare che volendo generalizzare il concetto di regolarit\`a presente nei politopi concreti
potremmo tentare pi\`u strade. Per esempio, definendo ricorsivamente il concetto di regolarit\`a, dicendo ossia che un politopo \`e regolare se sono
regolari le sue faccette e le figure al vertice (facendo quindi ricorsione sia dall'alto che dal basso sul poset-politopo) e definendo per esempio regolari
gli $0$-politopi, ossia i punti allora otterremmo la spiacevole situazione di scoprire tutti i politopi come regolari. Pertanto tale definizione non risulta
molto interessante da un punto di vista combinatorico.
Un altro tentativo - nella direzione che si \`e dimostrata essere quella giusta - potrebbe essere quello di definire come regolare un politopo che abbia le
facce indistinguibili, chiaramente a meno del rango delle stesse. Un politopo $\p$ sarebbe quindi regolare qualora il suo gruppo di simmetria
$\Gamma(\p)$ agisse transitivamente su $\p_i$ per ogni $i=-1,\dots,n$. Tuttavia questa definizione, pur restringendo considerevolmente l'insieme dei
politopi astratti, sembra non essere sufficiente per imbrigliare il concetto che abbiamo di politopo regolare.
Pertanto la definizione che \`e stata scelta di poligono regolare \`e leggermente pi\`u forte dell'uguaglianza tra le facce ed \'e quella dell'uguaglianza
tra le \emph{bandiere} del politopo. Descriviamo questo concetto con maggiore precisione.
\begin{defin}
Sia $\p$ un politopo di rango $n$. Diremo che $\p$ \`e \emph{regolare} se il suo gruppo di automorfismi $\Gamma(\p)$ agisce transitivamente sull'insieme
delle bandiere, $\mathcal{F}(\p)$.
\end{defin}
Notiamo come questa condizione di regolarit\`a sia chiaramente pi\`u forte della semplice richiesta dell'essenziale uguaglianza tra le facce di $\p$ a meno
del rango. In particolare quindi avremo che in un politopo regolare sia le faccette, quanto i vertici saranno \emph{indistinguibili}.

Veniamo adesso ad una fondamentale considerazione sui politopi regolari che completa la \ref{freeAction}.
\begin{prop}
Sia $\p$ un politopo regolare. Allora l'azione di $\Gamma(\p)$ \`e semplicemente transitiva sulle bandiere di $\p$. Ossia tale azione \`e 
libera e transitiva e in particolare, numericamente, avremo che
\begin{equation*}
\left|\Gamma(\p)\right|=\left|\mathcal{F}(\p)\right|
\end{equation*}
se $\p$ \`e finito
\end{prop}
\begin{proof}
Per \ref{freeAction} abbiamo che l'azione di $\Gamma(\p)$ su $\p$ \`e libera, quindi essendo $\p$ regolare per l'equazione degli stabilizzatori
nel caso in cui $\p$ sia finito, se $\Phi\in\mathcal{F}(\p)$ \`e una bandiera, avremo
\begin{equation*}
\left|\mathcal{F}(\p)\right|=\left|\Phi^{\Gamma(\p)}\right|=\left[\Gamma(\p):\Gamma(\p)_\Phi\right]=\left[\Gamma(\p):\{1\}\right]
=\left|\Gamma(\p)\right|
\end{equation*}
\end{proof}

Dimostriamo due semplici fatti che ci serviranno nell'immediato seguito per comprendere meglio le sezioni di un politopo regolare.

\begin{lem}
\label{lem:SectionFlags}
Sia $\p$ un $n$-politopo e $F/G$ una sua sezione, allora \`e definita la mappa di restrizione
\begin{gather*}
s:\mathcal{F}(\p)\longrightarrow\mathcal{F}(F/G)\\
\qquad\qquad \Phi\longmapsto\Phi\cap F/G=:\Phi_G^F
\end{gather*}
E tale mappa risulta suriettiva, ossia ogni bandiera della sezione $F/G$ proviene da una bandiera di $\p$
\end{lem}
\begin{proof}
Notiamo che $s$ \`e ben definita, infatti se $\Phi$ \`e una bandiera di $\p$, $\Phi_G^F$ risulta una catena di $F/G$
e se tale catena non fosse massimale, esisterebbe $H\in (F/G)\setminus\Phi_G^F$ tale che $\Phi\cup\left\{H\right\}$ risulti ancora essere
una catena (di $F/G$), ma allora $\Phi\cup\left\{H\right\}\supsetneq\Phi$, contro la massimalit\`a di $\Phi$.\\
Per mostrare la suriettivit\`a di $s$ \`e sufficiente notare che l'insieme delle catene di $\p$ \`e un insieme 
parzialmente ordinato (da $\subseteq$) \emph{induttivo} ossia
ogni catena di catene ha un elemento massimale, pertanto per il lemma di Zorn ogni catena \`e contenuta in una catena massimale.
Se allora $\Psi\in\mathcal{F}(F/G)$, abbiamo che $\Psi$ \`e una catena di $F/G$, pertanto esiste una $\Phi\in\mathcal{F}(\p)$
tale che $\Psi\subseteq\Phi$, da cui $s(\Phi)=\Psi$
\end{proof}

\`E bene notare che le regolarit\`a garantisce una enorme simmetria nel politopo, probabilmente la massima che abbia senso chiedere
in quanto ad uniformit\`a dello stesso. Questo fatto viene chiarito meglio con la seguente osservazione che ci fa naturalmente 
pensare a ci\`o che accade nel gruppo simmetrico $S_n$, considerando il modo in cui agisce su s\'e stesso.

\begin{oss}
\label{oss:ChainTransitivity}
Sia $\p$ un $n$-politopo regolare. Allora due catene $\mathit{C},\mathit{D}\subseteq\p$ sono coniugate tramite l'azione di $G:=\Gamma(\p)$ se
e solo se hanno lo stesso tipo.
\end{oss}
\begin{proof}
Ogni simmetria di $\p$ preserva il rango delle facce, pertanto chiaramente preserva il tipo delle catene. Mostriamo quindi il converso, ossia che se 
due catene sono dello stesso tipo, saranno coniugate per l'azione di $G$. Siano quindi $\mathit{C}$ e $\mathit{D}$ catene di $\p$ di tipo
$(i_1,\cdots,i_k)$. Troviamo due bandiere $\Phi_\mathit{C},\Phi_\mathit{D}\in\mathcal{F}(\p)$ tali che $\mathit{C}\subseteq\Phi_\mathit{C}$ e 
$\mathit{D}\subseteq\Phi_\mathit{D}$. Per la regolarit\`a di $\p$ esiste un $\sigma\in G$ tale che $\sigma\Phi_\mathit{C}=\Phi_\mathit{D}$
Notiamo quindi che
\begin{gather*}
\mathit{C}=\Phi_\mathit{C}\cap\left(\bigcup_{l=1}^k\p_{i_l}\right)\text{\quad e}\\
\mathit{D}=\Phi_\mathit{D}\cap\left(\bigcup_{l=1}^k\p_{i_l}\right)
\end{gather*}
Pertanto, applicando $\sigma$, avremo
\begin{gather*}
\sigma\left(\mathit{C}\right)=\sigma\left(\Phi_\mathit{C}\cap\left(\bigcup_{l=1}^k\p_{i_l}\right)\right)=\\
=\sigma\left(\Phi_\mathit{C}\right)\cap\sigma\left(\bigcup_{l=1}^k\p_{i_l}\right)=
\sigma\left(\Phi_\mathit{C}\right)\cap\left(\bigcup_{l=1}^k\sigma(\p_{i_l})\right)=\\
=\Phi_\mathit{D}\cap\left(\bigcup_{l=1}^k\p_{i_l}\right)=\mathit{D}
\end{gather*}
\end{proof}

Vediamo adesso che la regolarit\`a si preserva per sezioni e mostriamo il collegamento tra i gruppi di simmetria di un politopo e quelli delle sue sezioni.

\def\currentprefix{prop:RegularSections}
\begin{prop}
\label{prop:RegularSections}
Sia $\p$ un politopo regolare finito di rango $n$ e $G<F\in\p$ con $rkG=j$ e $rkF=i$. Allora
\begin{enumerate}
\item\locallabel{1} $F/G$ \`e regolare
\item\locallabel{2} Se $G'<F'\in\p$ sono tali che $rkF'=rkF$ e $rkG'=rkG$ allora $F'/G'\cong F/G$
\item\locallabel{3} Le faccette di $\p$ sono fra loro isomorfe
\item\locallabel{4} Le figure al vertice di $\p$ sono fra loro isomorfe
\item\locallabel{5} $\p$ \`e equivelare, possiede quindi un simbolo di Schl\"afli\\\\
%\end{enumerate}
Inoltre
%\begin{enumerate}
\item\locallabel{6} $\Gamma(F/G)\cong \Gamma(\p)_\Omega\leq\Gamma(\p)$ dove $\Omega$ \`e una catena di tipo $(-1,\dots,j,i,\dots,n)$ tale che $F,G\in\Omega$ ossia, in particolare
\item\locallabel{7} $\Gamma(F/G)\hookrightarrow\Gamma(\p)$
\end{enumerate}
\end{prop}

\begin{proof}
\point{1}Mostriamo che se $F/G$ \`e una sezione di $\p$, allora \`e anch'essa regolare. Siano quindi $\Psi_1,\Psi_2\in\mathcal{F}(F/G)$, essendo
bandiere di $F/G$ abbiamo che, come catene di $\p$, queste sono entrambe di tipo $(j,\cdots,i)$ e quindi per \ref{oss:ChainTransitivity} esiste
$\sigma\in\Gamma(\p)$ tale che $\sigma(\Psi_1)=\Psi_2$. Ora notiamo che $\sigma(F)=F$ e $\sigma(G)=G$, essendo $F$ e $G$ determinate
dai rispettivi ranghi sia in $\Psi_1$ che in $\Psi_2$. Da questo fatto per la monotonia di $\sigma$ abbiamo che $\sigma(F/G)=F/G$
(l'uguaglianza deriva dalla finitezza di $\p$). Pertanto effettivamente $\sigma\restrict{F/G}\in\Gamma(F/G)$
e ci\`o conclude la dimostrazione del primo punto.\\
\point{2}Se $F/G$ e $F'/G'$ sono sezioni di $\p$ con $rkF=rkF'=i$ e $rkG=rkG'=j$, allora $\left\{F,G\right\}$ e $\left\{F',G'\right\}$ sono entrambe
catene di tipo $(j,i)$ pertanto, per \ref{oss:ChainTransitivity}, esiste $\sigma\in\Gamma(\p)$ tale che $\sigma(F)=F'$ e 
$\sigma(G)=G'$ e per la monotonia di $\sigma$ abbiamo che $\sigma(F/G)=F'/G'$. pertanto
\begin{equation*}
F/G\xrightarrow{\quad\sigma\restrict{F/G}\quad}F'/G'
\end{equation*}
\`e un isomorfismo tra le due sezioni $F/G$ e $F'/G'$.\\
\point{3}Per il punto \localref{2} basta notare che una faccetta $F\in\p_{n-1}$ come politopo \`e per definizione $F/\bot$\\
\point{4}Nuovamente, per il punto \localref{2}, essendo le figure al vertice sezioni del tipo $\p/F$\\
\point{5}Sia $1\leq i\leq n-1$, se $F/G$ e $F'/G'$ sono sezioni con $F,F'\in\p_{i+1}$ e $G,G'\in\p_{i-2}$, allora
per il punto \localref{2} $F/G\cong F'/G'$, pertanto i numeri $p_i(F,G)$ e $p_i(F',G')$ coincidono, da cui l'quivelarit\`a di $\p$.\\
\point{6}L'isomorfismo in questione \`e chiaramente l'omomorfismo restrizione definito in \ref{prop:SectionGroups} che avevamo gi\`a mostrato
essere iniettivo. Non ci resta che mostrare che ogni simmetria di $F/G$ si pu\`o estendere a una simmetria del politopo $\p$. Sia, a tal proposito,
$\sigma\in\Gamma(F/G)$ e fissiamo una bandiera $\Psi_0\in\mathcal{F}(F/G)$. Allora $\Psi_0\cup\Omega$ \`e una bandiera di $\p$ ed, 
essendo $\sigma(\Psi_0)\in\mathcal{F}(F/G)$, abbiamo anche $\sigma(\Psi_0)\cup\Omega\in\flag{\p}$
Per la regolarit\`a di $\p$ abbiamo che esiste un $\tau\in\Gamma(\p)$ tale che
\begin{equation*}
\tau(\Psi_0\cup\Omega)=\sigma(\Psi_0)\cup\Omega
\end{equation*}
quindi $\tau(\Psi_0)=\sigma(\Psi_0)$ (in effetti, preservando sia $\tau$ che $\sigma$ i ranghi delle facce, avremo che
 $\tau\restrict{\Psi_0}=\sigma\restrict{\Psi_0}$).
Allora, ricordando che per \ref{freeAction} l'azione di $\Gamma(F/G)$ su $\flag{F/G}$ \`e libera, concludiamo che
 $\tau\restrict{F/G}=\sigma\restrict{F/G}$\\
 \point{7}Discende da \localref{6}
\end{proof}
\subsection{Generatori scelti per $\Gamma$}
Vediamo adesso un utile criterio per stabilire se un politopo \`e regolare esibendo particolari esempi di generatori che ci permetteranno quindi
di meglio studiare i gruppi di simmetrie dei politopi regolari. Cominciamo col definire i generatori scelti per una bandiera.
\begin{defin}
Sia $\Phi\in\flag{\p}$ una bandiera. Diremo che una $n$-upla $(\phi_i)_{i=0}^{n-1}$ \`e un sistema di generatori scelti relativi a $\Phi$ se
\begin{equation*}
\Phi^{\phi_i}=\Phi^i\qquad\forall i=0\,\dots,n-1
\end{equation*}
\end{defin}

Tali generatori che mostreremo derivare da ogni scelta di una bandiera di $\p$ avranno un ruolo essenziale per lo studio del gruppo $\Gamma(\p)$.
Vediamo quindi come la scelta della bandiera determini i diversi sistemi di generatori scelti.
\begin{oss}
\label{oss:GeneratingInvolutions}
Sia $\p$ un $n$-politopo, $\Phi,\Psi\in\flag{\p}$ due bandiere e $(\phi_i)_{i=0}^{n-1}, (\psi_i)_{i=0}^{n-1}$ loro rispettivi sistemi di
generatori scelti. Allora, se $\sigma\in\Gamma(\p)$ \`e tale che $\Phi^\sigma=\Psi$, i due sistemi di generatori scelti saranno anch'essi coniugati
da $\sigma$, ossia
\begin{equation*}
\phi_i^\sigma=\psi_i\quad\text{ per ogni }0\leq i\leq n-1
\end{equation*}
\end{oss}
\begin{proof}
Sia $0\leq i\leq n-1$, dobbiamo mostrare che $\Psi^{\phi_i^\sigma}=\Psi^i$, a tal fine dimostriamo che $\Psi^{\phi_i^\sigma}$ \`e
$i$-adiacente a $\Psi$ e da questa diversa. Sia infatti $G\in\Psi\setminus\p_i$ (una faccia di $\Psi$ non di rango $i$), allora
essendo $G^{\sigma^{-1}}\in\Phi\setminus\p_i$, quest'ultima faccia verr\`a fissata da $\phi_i$, pertanto
\begin{equation*}
G^{\phi_i^{\sigma}}=G^{\sigma^{-1}\phi_i\sigma}=G^{\sigma^{-1}\sigma}=G
\end{equation*}
Pertanto $\Psi^{\phi_i^\sigma}$ e $\Psi$ coincidono per tutti i ranghi diversi da $i$, per concludere ci basta osservare che se
coincidessero anche nel rango $i$-esimo, sarebbero uguali, ma da $\Psi^{\phi_i^\sigma}=\Psi$, per la libert\`a dell'azione di
$\Gamma$ sull'insieme delle bandiere concluderemmo $\phi_i^\sigma=1$, cio\`e $\phi_i=1$, assurdo per definizione del sistema di
generatori scelti relativi a $\Phi$. Pertanto abbiamo mostrato che $\Psi^{\phi_i^\sigma}$ \`e l'unica bandiera $i$-adiacente a $\Psi$
(diversa da $\Psi$ stessa) e, quindi, che $\phi_i^\sigma=\psi_i$
\end{proof}

\begin{prop}
\def\currentprefix{prop:IntroInvolutions}
\label{prop:IntroInvolutions}
Sia $\p$ un $n$-politopo. Allora sono equivalenti:
\begin{enumerate}
\item\locallabel{1}$\p$ \`e regolare
\item\locallabel{2}Esiste una bandiera $\Phi\in\flag{\p}$ che possiede un (unico) sistema di generatori scelti
\item\locallabel{3}Ogni bandiera $\Phi\in\flag{\p}$ possiede un (unico) sistema di generatori scelti
\end{enumerate}
\end{prop}
\begin{proof}
Notiamo che l'unicit\`a dei generatori scelti \`e in ogni caso garantita dalla libert\`a dell'azione di $\Gamma(\p)$ su $\flag{\p}$\\
\implication{1}{3}Se $\p$ \`e regolare e $\Phi\in\flag{\p}$ allora, ricordando che per $0\leq i\leq n-1\qquad\Phi^i$ \`e l'unica bandiera
$i$-adiacente a $\Phi$, esister\`a un $\sigma\in\Gamma(\p)$ tale che $\Phi^\sigma=\Phi^i$.\\
Osserviamo inoltre che se $\sigma$ \`e tale che $\Phi^\sigma=\Phi^i$ allora, rispettando gli automorfismi adiacenza e rango, abbiamo che
$\sigma$ deve mandare $\Phi^i$ necessariamente in $\Phi$, essendo quest'ultima l'unica bandiera adiacente a $\Phi^i$ rispetto al rango $i$
(non potendo essere $(\Phi^i)^\sigma=\Phi^i$ essendo $\sigma$ iniettivo). Pertanto
\begin{equation*}
\Phi^{\sigma^2}=(\Phi^\sigma)^\sigma=(\Phi^i)^\sigma=\Phi=\Phi^{id}
\end{equation*}
Pertanto, per la libert\`a dell'azione di $\Gamma(\p)$ su $\flag{\p}$, concludiamo che $\sigma^2=id$.\\
\implication{3}{2}Ovvio\\
\implication{2}{1}Supposto che valga $(3)$, sia $\Phi$ come da ipotesi e $\Psi\in\flag{\p}$ una bandiera. Mostriamo che
$\Psi$ \`e coniugata a $\Phi$ secondo l'azione di $\Gamma(\p)$.\\
Ricordiamo che, per \ref{prop:P3Flags}, $\p$ \`e (fortemente) connesso per bandiere, esiste pertanto una successione
$\Phi=\Upsilon_0,\cdots,\Upsilon_l=\Psi\in\flag{\p}$ di bandiere adiacenti da $\Phi$ a
$\Psi$. Mostriamo quindi per induzione su $l$ che esiste $\tau\in\Gamma(\p)$ tale che $\Phi^\tau=\Psi$\\
Se $l=0\quad\Phi=\Psi$ e basta prendere $\tau=id$. Se $l>0$, per ipotesi induttiva esiste un $\sigma\in\Gamma(\p)$ tale che
$\Phi^\sigma=\Upsilon_0^\sigma=\Upsilon_{l-1}$. Per ipotesi $\Upsilon_{l-1}$ e $\Upsilon_l$ sono $j$-adiacenti per un opportuno
$0\leq j\leq n-1$ quindi, se $(\upsilon_i)_{i=0}^{n-1}$ \`e il sistema di generatori relativi ad $\Upsilon_{l-1}$, per \ref{oss:GeneratingInvolutions}
 avremo che il sistema $(\upsilon_i)_{i=0}^{n-1}$ sar\`a il coniugato del sistema $(\phi_i)_{i=0}^{n-1}$ a mezzo di $\sigma$ quindi
\begin{equation*}
\Upsilon_l=\Upsilon_{l-1}^j=\Upsilon_{l-1}^{\upsilon_j}=\Upsilon^{\phi_j^\sigma}
\end{equation*}
pertanto $\Psi\in\Phi^(\Gamma(\p))$ che conclude la dimostrazione del passo induttivo e quindi della regolarit\`a di $\p$
\end{proof}

Alla luce di queste propriet\`a nel seguito se $\p$ sar\`a un $n$-politopo fisseremo una $\Phi\in\flag{\p}$ che chiameremo \emph{bandiera di base}
e - ove non specificamente indicato diversamente - indicheremo con $F_i$ la sua $i$-faccia e con $\phi_i\in\Gamma(\p)$ il suo $i$-esimo generatore
scelto.\\
Inoltre, se $J\subseteq\{0,\dots,n-1\}$ diremo i sottogruppi $\left\langle \phi_j\mid j\in J\right\rangle$ sottogruppi scelti. Pi\`u precisamente
 introdurremo le seguenti definizioni.
\begin{defin}
Sia $\p$ un $n$-politopo regolare e $\Phi\in\flag{\p}$ la sua bandiera (scelta) di base. Allora, se $J\subseteq\{0,\dots,n-1\}$ porremo
\begin{equation*}
\Phi_J=\left\{F_j\mid j\in J\right\}\subseteq\Phi
\end{equation*}
e
\begin{equation*}
\Gamma_J=\left\langle \sigma_i\mid i\notin J\right\rangle\leq \Gamma(\p)
\end{equation*}
\end{defin}
Vediamo quindi che tali sottogruppi scelti altro non sono che gli stabilizzatori in $\Gamma$ di \emph{parti} della bandiera di base $\Phi$.
\begin{prop}
\label{prop:DistSubgroups}
Sia $\p$ un politopo regolare, $\Phi$ la bandiera di base scelta. Allora per ogni $J\subseteq\{0,\dots,n-1\}$
\begin{equation*}
\Gamma_J=\Gamma(\p)_{\Phi_J}
\end{equation*}
\end{prop}
\begin{proof}
\point{\leq}\`E sufficiente notare che per ogni $i\notin J$ avremo che $\Phi_J^{\phi_i}=\Phi_J$, essendo $\Phi_J\subset\Phi$ e il rango $i$ non
presente nel tipo di $\Phi_J$\\
\point{\geq}Osserviamo che, posto
\begin{equation*}
X=\{\Psi\in\flag{\p}\mid \Psi\supseteq \Phi_J\}
\end{equation*}
$\Gamma_{\Phi_J}$ agisce in modo semplicemente transitivo su $X$ (ci\`o deriva dalla transitivit\`a dell'azione di $\Gamma(\p)$ su $\flag{\p}$).
Pertanto sar\`a sufficiente mostrare che $\Gamma_J\leq\Gamma_{\Phi_J}$ agisce anch'esso transitivamente su $X$ per concludere l'uguaglianza tra
i due sottogruppi.\\
Dimostriamo quindi che se $\Psi,\Upsilon\in X$ sono incidenti e $\Psi\in\Phi^{\Gamma_J}$ allora 
anche $\Psi$ sar\`a coniugata tramite l'azione di $\Gamma_J$ a $\Phi$. Siano pertanto
$\Psi,\Upsilon\in X$ e $k\notin J$ tale che $\Psi^k=\Upsilon$ e $\sigma\in\Gamma_J$ tale che $\Psi=\Phi^\sigma$.
Allora se $(\phi_i)_{i=1}^{n-1}$ e $(\psi_i)_{i=1}^{n-1}$ sono sistemi di involuzioni scelte
per $\Phi$ e $\Psi$ rispettivamente, per l'osservazione \ref{oss:GeneratingInvolutions} avremo che
\begin{gather*}
\Upsilon=\Psi^k=\Psi^{\psi_k}=\Phi^{\sigma\psi_k}=\Phi^{\sigma\phi_k^\sigma}=\Phi^{\phi_k\sigma}\in\Phi^{\Gamma_J}
\end{gather*}
\end{proof}
Alla luce di quanto visto possiamo ragionare per induzione. Sia $\Psi\in X$ e
\begin{equation*}
\Phi=\Phi_0,\cdots,\Phi_l=\Psi
\end{equation*}
una catena di bandiere adiacenti \emph{in X} da $\Phi$ a $\Psi$. Procediamo per induzione su $l$. Se $l=0$ non c'\`e nulla da dimostrare. 
Se $l>0$ per ipotesi induttiva $\Phi_{l-1}=\Phi^\sigma$ con $\sigma\in\Gamma_J$, pertanto per l'argomento precedente concludiamo che
$\Psi\in\Phi^{\Gamma_J}$.\\
Per la genericit\`a di $\Psi\in X$ abbiamo mostrato la transitivit\`a dell'azione di $\Gamma_J$ su $X$ e, pertanto, $\Gamma_J=\Gamma_{\Phi_J}$
A questo punto possiamo finalmente fornire una prima presentazione del gruppo di simmetrie di un politopo regolare. Dimostreremo infatti che i
generatori scelti di cui abbiamo parlato sono effettivame dei generatori per il gruppo delle simmetrie del politopo. Pertanto ogni scelta di una bandiera
fornir\`a una presentazione per il gruppo di simmetrie.\\
\begin{teo}
\label{teo:Generators}
Sia $\p$ un $n$-politopo, $\Phi$ una sua bandiera e $(\phi_i)_{i=0}^{n-1}$ il suo sistema di involuzioni scelte. Allora
\begin{equation*}
\Gamma(\p)=\langle\phi_0,\cdots,\phi_{n-1}\rangle
\end{equation*}
\end{teo}
\begin{proof}
Per dimostrare l'asserto \`e sufficiente applicare la proposizione \ref{prop:DistSubgroups} nel caso particolare $J=\emptyset$
\end{proof}
Vediamo adesso un semplice esempio di un caso particolare: i politopi regolari di rango $2$, cio\`e i poligoni regolari. Ne calcoleremo
i gruppi di simmetria mostrando che vale ci\`o che sappiamo dal caso concreto, ossia tali gruppi risultano essere i gruppi diedrali $\mathcal{D}_n$.
Utilizzeremo a tal fine un semplice lemma che sostanzialmente ci permette di utilizzare una presentazione dei gruppi diedrali pi\'u comoda nel
nostro contesto.
\begin{lem}
\label{lem:DihedralPresentations}
Sia $G=\langle x,y\rangle$ un gruppo finito generato dalle due \emph{involuzioni} $x$ e $y$. Allora, posto $p=o(xy)$, avremo che
\begin{equation*}
G=\langle \sigma,\rho\mid \sigma^2=\rho^p=1,\rho^\sigma=\rho^{-1}\rangle\cong\mathcal{D}_p
\end{equation*}
dove $\rho=xy$ e $\sigma=x$
\end{lem}
\begin{proof}
Osserviamo che ha senso considerare solamente il caso in cui $x\neq y$. Posto $\rho=xy$, notiamo che
\begin{equation*}
\rho^x=(xy)^x=xxyx=yx=(xy)^{-1}
\end{equation*}
Da cui $H:=\langle\rho\rangle\trianglelefteq G$ e la struttura di $G$ risulta pertanto determinata dall'azione di $\langle x\rangle$ sul
sottogruppo ciclico $H$ (di ordine $p$). Quindi
\begin{equation*}
G=\langle x\rangle\ltimes H\cong\mathcal{D}_p
\end{equation*}
\end{proof}
Passiamo quindi ora all'esempio fondamentale dei poligoni.
\begin{example}
\label{example:PolygonsDihedral}
Sia $P$ un poligono (astratto) regolare e $p=\left|P_1\right|$ il numero dei suoi lati (faccette). Allora
\begin{equation*}
\Gamma(P)=\langle x,y\rangle\cong\mathcal{D}_p
\end{equation*}
dove $x$ e $y$ sono rispettivamente i generatori scelti $\phi_0,\phi_1$ relativi alla bandiera (propria) di base $\Phi_0=\{A_0,l_0\}$ e
$\mathcal{D}_p=\langle \sigma,\rho\mid \rho^p=\sigma^2=1,\rho^\sigma=\rho^{-1}\rangle$ \`e il $p$-esimo gruppo diedrale.
\end{example}
\begin{proof}
Per il teorema \ref{teo:Generators}, abbiamo che $\Gamma=\langle x,y\rangle$. Inoltre $\Gamma$ agisce transitivamente su $P_1$, l'insieme dei
lati di $P$. Se $l\in P_1$, per la proposizione \ref{prop:DistSubgroups} abbiamo che $\Gamma_l=\Gamma(l/\bot)\cong\langle x\rangle$
pertanto, passando alle cardinalit\`a:
\begin{equation*}
p=\left|P_1\right|=\left|l^\Gamma\right|=\left[\Gamma:\Gamma_l\right]=\frac{\left|\Gamma\right|}{\left|\Gamma_l\right|}
\end{equation*}
Essendo $\Gamma_l\cong\Gamma(l/\bot)$. Ma per il lemma \ref{lem:DihedralPresentations} abbiamo che $\Gamma(P)$ \`e un gruppo diedrale $\mathcal{D}_n$
quindi concludiamo con le cardinalit\`a:
\begin{equation*}
2p=\left|\Gamma(P)\right|=\left|\mathcal{D}_n\right|=2n
\end{equation*}
pertanto $\Gamma(P)\cong\mathcal{D}_p$, con $p$ numero di lati di $P$
\end{proof}
Vediamo adesso come tali sottogruppi scelti ci aiutino a dare una presentazione dei gruppi di simmetria delle sezioni del politopo regolare $\p$ tra
cui i casi importanti delle faccette, delle figure al vertice e delle sezioni poligonali (ossia le $2$-sezioni). Tali strutture infatti avranno gruppi
di simmetria che si descrivono facilmente a mezzo dei generatori scelti, essendo questi alcuni sottogruppi scelti.
\def\currentprefix{prop:SelectedSubgroupsSections}
\begin{prop}
\label{prop:SelectedSubgroupsSections}
Sia $\p$ un $n$-politopo regolare (con $\Phi$ bandiera di base e $(\phi_i)_{i=0}^{n-1}$ i suoi generatori scelti),
$F,G\in\p$ due facce di $\p$ con $G<F$ con $rkF=i$ e $rkG=j$. Allora:
\begin{enumerate}
\item\locallabel{1}$\Gamma(F/G)\cong\langle \phi_{j+1},\cdots,\phi_{i-1}\rangle$
\item\locallabel{2}$\Gamma(F/\bot)\cong\langle \phi_0,\cdots,\phi_{i-1}\rangle$
\item\locallabel{3}$\Gamma(\p/F)\cong\langle\phi_{i+1},\cdots,\phi_{n-1}\rangle$
\item\locallabel{4}Se $rkF=k+1$ e $rkG=k-2$ e $\p$ \`e di tipo $\left\{p_1,\cdots,p_{n-1}\right\}$ allora
\begin{equation}
\Gamma(F/G)\cong\mathcal{D}_{p_k}
\end{equation}
\end{enumerate}
\end{prop}
\begin{proof}
\point{1}Siano $F$ e $G$ come da ipotesi. Posto $J=\{0,\dots,n-1\}$, dobbiamo mostrare che $\Gamma(F/G)\cong\Gamma_J$, pertanto notiamo che per
\ref{prop:RegularSections} $\Gamma(F/G)\cong\Gamma(\p)_\Omega$ se $\Omega$ \`e una catena in $\p$ di tipo $J$, in particolare essendo per costruzione
$\Phi_J$ di tipo $J$ avremo che $\Gamma(F/G)\cong\Gamma_J$. Concludiamo osservando che, per \ref{prop:DistSubgroups}, $\Gamma_J\cong\Gamma(\p)_{\Phi_J}$
da cui $\Gamma(F/G)\cong\Gamma_J$\\
$\bm{(2)}$,\point{3} Discendono da \localref{1}\\
\point{4}Se $\p$ ha simbolo $\{p_1,\cdots,p_{n-1}\}$ allora la sezione $F/G$ - essendo $rkF=k+1$ e $rkG=k-1$ - \`e un 
$p_k$-agono. Ora, da \localref{1} abbiamo che $\Gamma(F/G)\cong\langle\phi_{k-1},\phi_k\rangle$, applichiamo quindi l'esempio \ref{example:PolygonsDihedral}
per concludere che $\Gamma(F/G)\cong\mathcal{D}_{p_k}$.
\end{proof}
Proseguiamo nello studio del gruppo $\Gamma$ vedendo un'altra utile propriet\`a dei suoi sottogruppi scelti $\Gamma_J$ (relativi alla scelta di una
bandiera di base per $\p$). Questi infatti costituiscono una famiglia di sottogruppi di $\Gamma$ chiusa per intersezioni. Nel seguito per comodit\`a
denoteremo con $N=\{0,\dots,n-1\}$ l'insieme massimo di indici di famiglie proprie dell'$n$-politopo regolare $\p$.
\begin{prop}[Propriet\`a dell'intersezione]
\label{prop:IntersectionProperty}
Se $\p$ regolare con bandiera di base $\Phi$ allora la famiglia $\mathcal{S}_\Phi=\{\Gamma_J\mid J\subseteq N\}$ \`e chiusa per intersezioni.\\
Pi\`u precisamente, se $I,J\subseteq N$, allora
\begin{equation*}
\Gamma_I\cap\Gamma_J=\Gamma_{I\cup J}
\end{equation*}
\begin{proof}
L'asserto deriva dalla caratterizzazione dei sottogruppi scelti come stabilizzatori di particolari sottocatene di $\Phi$. Infatti
se $I,J\subseteq N$, per \ref{prop:DistSubgroups}, $\Gamma_I=\Gamma(\p)_{\Phi_I}$ e $\Gamma_J=\Gamma(\p)_{\Phi_J}$, gli stabilizzatori
in $\Gamma(\p)$ di $\Phi_I$ e $\Phi_J$ rispettivamente. pertanto concludiamo notando che
\begin{equation*}
\Gamma(\p)_{\Phi_I}\cap\Gamma(\p)_{\Phi_J}=\Gamma(\p)_{\Phi_I\cup\Phi_J}=\Gamma(\p)_{\Phi_{I\cup J}}=\Gamma_{I\cup J}
\end{equation*}
per la definizione delle catene $\Phi_I,\Phi_J$
\end{proof}
\end{prop}
Proseguiamo con lo studio dei generatori scelti rispetto ad una bandiera di base osservando che questi - a patto che siano di indici
non conseguenti - commutano, cosa che semplifica notevolmente lo studio del gruppo delle simmetrie di un politopo regolare.
\begin{prop}
\label{prop:CommutatingGenerators}
Sia $\p$ un $n$-politopo regolare con bandiera di base $\Phi$ e $(\phi_l)_{l=0}^{n-1}$ generatori scelti. Allora
\begin{equation*}
|i-j|\geq 2\Longrightarrow \phi_i\phi_j=\phi_j\phi_i
\end{equation*}
\end{prop}
\begin{proof}
Sia $\mathcal{C}=\Phi_{N\setminus\{i,j\}}$ la catena che si ottiene escludendo da $\Phi=\{F_l\}_{l=-1}^n$ le facce di rango $i$ e $j$. Vediamo che tale
catena ha precisamente $4$ completamenti a bandiere di $\p$, notiamo intanto che essa ha tipo
\begin{equation*}
(-1,\cdots,j-1,j+1,\cdots,i-1,i+1,\cdots,n)
\end{equation*}
Ora, sia la sezione $F_{i+1}/F_{i-1}$ che la sezione $F_{j+1}/F_{j-1}$ - essendo di rango due - hanno esattamente due facce proprie, ognuna avendone una
in $\Phi$, chiamiamo quindi le altre due $F'_i$ e $F'_j$ rispettivamente. Allora le estensioni di $\mathcal{C}$ a bandiera di $\p$ saranno esattamente
\begin{gather*}
\mathcal{C}_0:=\mathcal{C}\cup\{F_j,F_i\}(=\Phi)\\
\mathcal{C}_1:=\mathcal{C}\cup\{F_j',F_i\}(=\Phi^{\phi_j})\\
\mathcal{C}_2:=\mathcal{C}\cup\{F_j,F_i'\}(=\Phi^{\phi_i})\\
\mathcal{C}_3:=\mathcal{C}\cup\{F_j',F_i'\}
\end{gather*}
A questo punto, essendo l'azione sulle bandiere di $\Gamma$ transitiva, avremo che le $4$ bandiere che contengono $\mathcal{C}$ saranno coniugate
tramite $\Gamma$ e - pi\`u precisamente - $\Gamma_\mathcal{C}$. Ricordiamo che $\Gamma_\mathcal{C}\cong\langle\phi_i,\phi_j\rangle$ per
la proposizione \ref{prop:DistSubgroups}. Si ha pertanto per la libert\`a dell'azione di $\Gamma_\mathcal{C}$ su
$\Phi^{\Gamma_\mathcal{C}}=\{C_0,C_1,C_2,C_3\}$ che
\begin{equation*}
|\langle\phi_i,\phi_j\rangle|=|\Gamma_\mathcal{C}|=\left[\Gamma_\mathcal{C}:(\Gamma_\mathcal{C})_\Phi\right]=\left|\Phi^{\Gamma_\mathcal{C}}\right|=4
\end{equation*}
Questo ci dice che le due involuzioni (distinte) $\phi_i$ e $\phi_j$ generano un gruppo di ordine $4$, pertanto isomorfo al gruppo abeliano $C_2^2$.
\end{proof}
Fancendo il punto della situazione abbiamo visto che il gruppo $\Gamma$ \`e un gruppo generato da involuzioni $(\phi_i)_{i=0}^{n-1}$ tali che
se $|i-j|\geq 2$ allora $\phi_i$ e $\phi_j$ commutano. Un gruppo di tal tipo \`e un cosiddetto \emph{string involution generated group},
abbreviato in \emph{sggi}. Inoltre il punto $4$ di \ref{prop:SelectedSubgroupsSections} ci garantisce che se commutano le ``vicine''
involuzioni $\phi_{j-1}$ e $\phi_j$ allora necessariamente avremo $p_j=2$ (generando queste il $p_j$-esimo gruppo diedrale, che sar\`a per\`o isomorfo
a $C_2^2$, essendo $\phi_{j-1}\phi_j=\phi_j\phi_{j-1}$). Situazione che vedremo - nel seguito - come sostanzialmente degenere nel campo dei
politopi regolari.\\\\
Continuiamo lo studio di $\Gamma$ e in particolare degli stabilizzatori delle facce di $\p$ (particolari sottogruppi scelti).
Tali stabilizzatori si spezzano infatti come prodotto di loro sottogruppi normali.
Possiamo interpretare la cosa come - una volta scelta una faccia $F$ - una decomposizione dell'azione
dello stabilizzatore di $F$ nella parte \emph{sopra} la faccia e nella parte \emph{sotto} (sostanzialmente la faccia stessa), pi\`u concisamente
possiamo dire
che $\Gamma_F$ sar\`a rappresentato (fedelmente) dalle due azioni sulla faccia $F$ (pi\`u propriamente $F/\bot$) e sulla sua co-faccia, $\p/F$.
\begin{lem}
\label{lem:StabilizerFaceIsProduct}
Sia $F\in\p_i\cap\Phi$ allora
\begin{equation*}
\Gamma_F=\Gamma_{i}\cong\langle\phi_0,\cdots,\phi_{i-1}\rangle\times\langle\phi_{i+1},\cdots,\phi_{n-1}\rangle\cong\Gamma(F/\bot)\times\Gamma(\p/F)
\end{equation*}
Ossia, le simmetrie che fissano una faccia di $\p$ si scompongono nella loro azione sulle facce sopra $F$ e nell'azione su $F$ stessa.
\end{lem}
\begin{proof}
Per \ref{prop:DistSubgroups} abbiamo che 
\begin{equation*}
\Gamma_F=\Gamma_{\Phi_i}\cong\langle\phi_j\mid j\neq i\rangle
\end{equation*}
Ma, per la propriet\`a dell'intersezione, \ref{prop:IntersectionProperty}, quest'ultimo gruppo si spezza come il prodotto
$\langle\phi_j\mid j<i\rangle\times\langle\phi_j\mid j>i\rangle$ ma questi due fattori sono effettivamente i gruppi di simmetria
della faccia $F$ e della sua co-faccia $\p/F$ rispettivamente, infatti per \ref{prop:SelectedSubgroupsSections}, concludiamo che
\begin{equation*}
\langle\phi_j\mid j<i\rangle\times\langle\phi_j\mid j>i\rangle\cong\Gamma(F/\bot)\times\Gamma(\p/F)
\end{equation*}
da cui la tesi.
\end{proof}
Proseguiamo con un lemma che caratterizza l'incidenza delle facce mediante il gruppo di simmetria e le facce della bandiera scelta. Quindi ricordiamo che
nel seguito $\Phi=\{F_i\}_{i=0}^{n-1}$ sar\`a la bandiera scelta e $F_i$ le sue facce.
\begin{lem}
\label{lem:IncidenceSelectedFaces}
Sia $G\in\p_j$. Allora $G$ \`e incidente a $F_i$ in $\p$ se e solo se questa - a meno dell'azione di
$\Gamma_i$ - lo \`e in $\Phi$. Ossia se
\begin{equation*}
G^{\Gamma_i}=F_j^{\Gamma_i}
\end{equation*}
\end{lem}
\begin{proof}
Supponiamo $G<F_i$ o $G>F_i$. In entrambi i casi sia ${G,F_i}\subseteq\Psi\in\flag{\p}$ e - per la regolarit\`a di $\p$, $\sigma\in\Gamma$ tale che
$\Psi^\sigma=\Phi$, allora avremo che $G^\sigma=F_j$ e $F_i^\sigma=F_i$, pertanto $G$ e $F_j$ sono coniugati tramite un elemento
$\sigma\in\Gamma_i=\Gamma_{\Phi_\{i\}}$.\\
Supponiamo adesso che esista un $\sigma\in\Gamma_i$ tale che $G^\sigma=F_j$. Allora, essendo $\sigma^{-1}\in\Gamma_i$ un automorfismo di $\p$ che fissa
$F_i$ avremo che la faccia $G=(G^\sigma)^{\sigma^{-1}}=F_j^{\sigma^{-1}}$ \`e incidente a $F_i^{\sigma^{-1}}=F_i$.
\end{proof}
Utilizziamo questo lemma per dimostrare finalmente come la struttura combinatoria di $\p$ sia sostanzialmente codificata nella struttura
esclusivamente algebrica del gruppo $\Gamma=\Gamma(\p)$ (a meno della scelta originale di una bandiera e quindi dei conseguenti generatori scelti).
Infatti il seguente teorema mostra come l'incidenza di due qualsiasi facce di $\p$ sia equivalente a due altre propriet\`a esclusivamente algebriche.
Enunciamo e dimostriamo il teorema in questione. [FORSE SERVIREBBE DEFINIRLI PRIMA]Ma prima diamo una definizione per semplificare la notazione riguardo
sottogruppi scelti in alcuni punti, con una conseguente osservazione.
\begin{defin}
Se $\Phi$ \`e la bandiera scelta di $\p$, $(\phi_i)_{i=0}^{n-1}$ \`e il suo sistema di generatori scelti, e $-1\leq j,k\leq n$ definiamo
\begin{equation*}
\Gamma_{j\rightarrow k}:=\langle\phi_{j+1},\cdots,\phi_{k-1}\rangle=\Gamma_{\{j+1,\dots,k-1\}}
\end{equation*}
se $k-j\geq 2$ e
\begin{equation*}
\Gamma_{j\rightarrow k}:=\{1\}
\end{equation*}
altrimenti
\end{defin}
\begin{oss}
\label{oss:SectionSubgroupNotation}
Se $-1\leq i,j\leq n$ allora
\begin{equation*}
\Gamma_{i\rightarrow j}\cong\Gamma(F_i/F_j)
\end{equation*}
\end{oss}
\begin{proof}
\`E un'essenziale riformulazione di \ref{prop:SelectedSubgroupsSections}
\end{proof}
Inoltre nel prossimo teorema ci saranno utili le due seguenti osservazioni.
\begin{oss}
\label{oss:SectionSubgroupCommuting}
Siano $0\leq j,k,l,m\leq n$. Se $k\leq l$ allora
\begin{equation*}
\left[\Gamma_{j\rightarrow k},\Gamma_{l\rightarrow m}\right]=\{1\}
\end{equation*}
Dove il termine a sinistra \`e il sottogruppo commutatore tra i due sottogruppi. Abbiamo cio\`e che i due sottogruppi
commuteranno puntualmente e in particolare $\Gamma_{j\rightarrow k}\Gamma_{l\rightarrow m}=\Gamma_{l\rightarrow m}\Gamma_{j\rightarrow k}$
che risulter\`a quindi un sottogruppo di $\Gamma$ (cosa altrimenti non garantita)
\end{oss}
\begin{proof}
Nei casi banali $j\geq k$ o $m\geq l$ uno dei due sottogruppi risulter\`a il sottogruppo identico rendendo banale l'asserto. Se, invece, gli indici
sono nell'ordine
\begin{equation*}
0\leq j\leq k\leq l\leq m\leq n
\end{equation*}
Ricordando che
\begin{gather*}
\Gamma_{j\rightarrow k}=\langle\phi_{j+1},\dots\phi_{k-1}\rangle\text{ e }\\
\Gamma_{l\rightarrow m}=\langle\phi_{l+1},\dots\phi_{m-1}\rangle
\end{gather*}
Ogni generatore di $\Gamma_{j\rightarrow k}$ commuter\`a con ogni generatore di $\Gamma_{l\rightarrow m}$ (per \ref{prop:CommutatingGenerators}), essendo
\begin{equation*}
l+1-(k-1)=l-k+2\geq 2
\end{equation*}
\end{proof}
La seguente osservazione \`e essenzialmente una conseguenza immediata della definizione dei sottogruppi del tipo $\Gamma_{i\rightarrow j}$
e dell'osservazione \ref{oss:SectionSubgroupCommuting}, essendo un caso particolare di quest'ultima.
[E FORSE \`E INUTILE SE INTRODUCIAMO PRIMA LA DEFINIZIONE DEI SOTTOGRUPPI SEZIONE]
\begin{oss}
\label{oss:StabilizerFactorsAsSectionSubgroups}
\begin{itemize}
\item Se $0\leq j\leq n-1$ allora
\begin{equation*}
\Gamma_j\cong\Gamma_{-1\rightarrow j}\Gamma_{j\rightarrow n}=\Gamma_{j\rightarrow n}\Gamma_{-1\rightarrow j}
\end{equation*}
\item Se $-1\leq j\leq j'\leq i'\leq i\leq n$ allora
\begin{equation*}
\Gamma_{i'\rightarrow j'}\leq\Gamma_{i\rightarrow j}
\end{equation*}
\end{itemize}
\end{oss}
\begin{proof}
Banale
\end{proof}

Possiamo ora pi\`u agevolmente procedere con la dimostrazione del teorema.
\begin{teo}
\label{teo:AlgebraicIncidence}
[NEL LIBRO SBAGLIATO, PER ME]
\def\currentprefix{teo:AlgebraicIncidence}
\locallabel{teo:AlgebraicIncidence}
Siano $0\leq j\leq i\leq n-1$ (essendo $\Phi=\{F_i\}_{i=0}^{n-1}$ la bandiera di base e $(\phi_i)_{i=0}^{n-1}$ i suoi generatori scelti) 
e $\sigma,\tau\in\Gamma$. Allora
sono equivalenti le seguenti:
\begin{enumerate}
\item\locallabel{1}$F_j^\sigma\leq F_i^\tau$
\item\locallabel{2}$\Gamma_j\sigma\cap\Gamma_i\tau\neq\emptyset$
\item\locallabel{3}$\sigma\tau^{-1}\in\Gamma_{j\rightarrow n}\Gamma_{-1\rightarrow i}$
\end{enumerate}
\end{teo}
\begin{proof}
\implication{1}{2}Se $F_j^\sigma\leq F_i^\tau$ allora $F_j^{\sigma\tau^{-1}}\leq F_i$ pertanto, alla luce del lemma \ref{lem:IncidenceSelectedFaces},
avremo che $F_j^{\sigma\tau^{-1}}=F_j^\gamma$ per un opportuno $\gamma\in\Gamma_i$, perci\`o
\begin{equation*}
F_j^{\sigma\tau^{-1}\gamma^{-1}}=F_j
\end{equation*}
da cui
\begin{equation*}
\sigma\tau^{-1}\gamma^{-1}=:\rho\in\Gamma_j
\end{equation*}
si ha pertanto che
\begin{equation*}
\Gamma_i\ni\gamma=\rho^{-1}\sigma\tau^{-1}\in\Gamma_j\sigma\tau^{-1}
\end{equation*}
Quindi $(\Gamma_j\sigma\tau^{-1})\cap\Gamma_i\neq\emptyset$ e (applicando $\tau$) anche $\Gamma_j\sigma\cap\Gamma_i\tau\neq\emptyset$.\\
\implication{2}{3}Essendo $\Gamma_j\phi\psi^{-1}\cap\Gamma_i\neq\emptyset$, prendiamo quindi
\begin{equation*}
\gamma=\rho\phi\psi^{-1}\in\Gamma_i\quad\text{con }\rho\in\Gamma_j
\end{equation*}
Pertanto avremo che $\phi\psi^{-1}=\rho^{-1}\gamma\in\Gamma_j\Gamma_i$ per avere (\localref{3}) ci baster\`a mostrare che
\begin{equation*}
\Gamma_j\Gamma_i=\Gamma_{j\rightarrow n}\Gamma_{-1\rightarrow i}
\end{equation*}
Quindi, grazie all'osservazione \ref{oss:StabilizerFactorsAsSectionSubgroups}, abbiamo che
\begin{gather*}
\Gamma_j\Gamma_i=\Gamma_{-1\rightarrow j}\Gamma_{j\rightarrow n}\Gamma_{-1\rightarrow i}\Gamma_{i\rightarrow n}=
\Gamma_{j\rightarrow n}\Gamma_{-1\rightarrow j}\Gamma_{-1\rightarrow i}\Gamma_{i\rightarrow n}=\\
=\Gamma_{j\rightarrow n}\Gamma_{-1\rightarrow i}\Gamma_{i\rightarrow n}=\Gamma_{j\rightarrow n}\Gamma_{i\rightarrow n}\Gamma_{-1\rightarrow i}=
\Gamma_{j\rightarrow n}\Gamma_{-1\rightarrow i}
\end{gather*}
[EVENTUALMENTE AGGIUNGERE MOTIVAZIONI SUGLI UGUALI]
\implication{3}{1}Supponiamo che $\sigma\tau^{-1}=\rho\gamma$ con $\rho\in\Gamma_{j\rightarrow n}$ e $\gamma\in\Gamma_{-1\rightarrow i}$. Allora
ricordando \ref{lem:IncidenceSelectedFaces} avremo che
\begin{equation*}
F_j^{\sigma\tau^{-1}}=F_j^{\rho\gamma}=F_j^\gamma\leq F_i^\gamma=F_i
\end{equation*}
\end{proof}
Questo teorema ci permette di caratterizzare l'incidenza delle facce di un politopo regolare semplicemente dal suo gruppo di simmetria, in effetti
abbiamo mostrato che due facce $F_i^\phi$ e $F_j^\psi$ sono incidenti precisamente quando si intersecano le loro relative classi laterali (destre)
$\Gamma_i\phi$ e $\Gamma_j\psi$.\\
Dopo aver introdotto e compreso un po' dei generatori scelti del politopo regolare $\p$ vediamo quali relazioni fra loro intercorrono, per arrivare
quindi ad una presentazione del gruppo $\Gamma$.
\subsection{Relazioni}
Abbiamo visto che un politopo regolare \`e in particolare univelare, possiede cio\`e un \emph{tipo} (o simbolo di Shl\"afli). Tale simbolo,
che descrive la ``forma'' locale del politopo ad ogni livello (dicendoci a che poligono corrispondono le sezioni di rango $2$), unitamente
alla regolarit\`a di $\p$ ci d\`a importantissime informazioni sulla struttura del gruppo di simmetria di $\p$, $\Gamma$. Inoltre tale
informazione - a sua volta - descriver\`a il politopo stesso, mostrandoci cos\`i come esso possa essere visto sia da un punto di vista
di insieme parzialmente ordinato, che da un punto di vista squisitamente algebrico, visualizzando come facce particolari classi laterali
del gruppo $\Gamma$.\\
Vediamo quindi come il tipo di $\p$ determini in modo diretto delle relazioni tra i generatori scelti. \`E bene notare tuttavia che continueremo
a presupporre la scelta di una bandiera di base in $\flag{\p}$.
\begin{teo}
Sia $\p$ un politopo regolare di tipo $\{p_1,\cdots,p_{n-1}\}$ con generatori scelti $\{\phi_i\}_{i=0}^{n-1}$ allora per $0\leq j\leq i\leq n-1$
tali generatori soddisferanno le seguenti relazioni
\begin{equation*}
(\phi_i\phi_j)^{p_{ij}}=1
\end{equation*}
dove gli ordini $p_{ij}$ sono dati da
\begin{equation*}
p_{ij}=\begin{cases}
			1	&	\text{se }i=j\\
			p_i	&	\text{se }i=j+1\\
			2	&	\text{se, altrimenti, }i\geq j+2
		\end{cases}
\end{equation*}
\begin{proof}
Abbiamo in realt\`a gi\`a dimostrato tale teorema, vediamo perch\'e. Siano infatti $0\leq j\leq i\leq n-1$ e consideriamo i tre casi:\\
\point{i=j}Il caso in cui $i=j$ \`e semplicemente una riformulazione del fatto che i generatori scelti siano involuzioni.\\
\point{i=j+1}Grazie a \ref{prop:SelectedSubgroupsSections} abbiamo che $\langle\phi_i,\phi_j\rangle\cong\mathcal{D}_i$
pertanto $\phi_i$ e $\phi_j$ sono due involuzioni generanti il sottogruppo diedrale di cardinalit\`a $2i$ e $\phi_i\phi_j$ genera
il suo sottogruppo delle rotazioni, isomorfo a $C_i$, l'$i$-esimo gruppo ciclico.\\
\point{i\geq j+2}In questo caso \`e sufficiente applicare la proposizione \ref{prop:CommutatingGenerators} per concludere che
$\phi_i\phi_j=\phi_j\phi_i$ e quindi che
\begin{equation*}
(\phi_i\phi_j)^2=\phi_i\phi_j\phi_i\phi_j=\phi_i^2\phi_j^2=1
\end{equation*}
\end{proof}
\end{teo}
Grazie a questo teorema abbiamo mostrato che $\Gamma(\p)$, se $\p$ \`e un $n$-politopo regolare di tipo $\{p_1,\cdots,p_{n-1}\}$, \`e un quoziente
del gruppo stringa di Coxeter di tipo $\{p_1,\cdots,p_{n-1}\}$. Parleremo dei gruppi di Coxeter stringa nel seguito. [O FORSE \`E MEGLIO ACCENNARE
E AGGIUNGERE UN DIAGRAMMA?]\\
Vediamo adesso che il caso in cui uno dei numeri nel simbolo di Shl\"afli di $\p$ sia $2$ \`e molto particolare, quasi patologic (tanto da non
verificarsi nel caso di politopi regolari concreti). Quando ci\`o capita ad un certo rango, le facce di tale rango e di quello precedente sono
tutte incidenti.
\begin{teo}
\label{teo:ReducibleDiagram}
\def\currentprefix{teo:ReducibleDiagram}
Sia $\p$ (regolare) di tipo $\{p_1,\cdots,p_{n-1}\}$ e sia $1\leq k\leq n-1$, allora sono equivalenti le seguenti:
\begin{enumerate}
\item\locallabel{1}$p_k=2$
\item\locallabel{2}per ogni $G\in\p_{k-1}$ e $F\in\p_k$ si ha $G<F$
\end{enumerate}
Inoltre, quando si verificano tali condizioni, si ha anche che $\Gamma$ si spezza in prodotto diretto dei suoi sottogruppi di sezione
a livello $k$, ossia
\begin{equation}
\locallabel{3}
\Gamma=\Gamma_{-1\rightarrow k}\times\Gamma_{k-1\rightarrow n}=\langle\phi_0,\cdots,\phi_{k-1}\rangle\times\langle\phi_k,\cdots,\phi_{n-1}\rangle\tag{3}
\end{equation}
\end{teo}
\begin{proof}
$\bm{(1)\Rightarrow(2)\land(\localref{3})}$\textbf{: }Essendo $p_k=2$ abbiamo (come sopra) che $\left[\phi_{k-1},\phi_k\right]=1$, inoltre essendo in
generale, per \ref{prop:CommutatingGenerators}, $\left[\phi_i,\phi_j\right]=1$ per ogni $0\leq i,j\leq n-1$ con $\left|i-j\right|\geq 2$, avremo che
\begin{equation}
\locallabel{commutano}
\left[\phi_i,\phi_j\right]=1\quad\forall 0\leq j<k\leq i\leq n-1
\end{equation}
Ossia, i generatori scelti relativi ai ranghi pi\`u piccoli di $k$ commuteranno con quelli di ranghi pi\`u grandi di $k$.
Pertanto, essendo $\Gamma=\langle\phi_0,\cdots,\phi_{n-1}\rangle$, grazie a queste relazioni di commutativit\`a, abbiamo
\begin{gather}
\locallabel{generano}
\Gamma=\langle\phi_0,\cdots,\phi_{k-1}\rangle\langle\phi_k,\cdots,\phi_{n-1}\rangle=\\
=\Gamma_{-1\rightarrow k}\Gamma_{k-1\rightarrow n}=\Gamma_{k-1\rightarrow n}\Gamma_{-1\rightarrow k}
\end{gather}
Siano a questo punto $G\in\p_{k-1}$ e $F\in\p_k$, allora esisteranno $\sigma,\tau\in\Gamma$ tali che $G=F_{k-1}^\sigma$ e $F=F_i^\tau$ e utilizzando
il teorema \ref{teo:AlgebraicIncidence}, essendo ovviamente $\sigma\tau^{-1}\in\Gamma=\Gamma_{k-1\rightarrow n}\Gamma_{-1\rightarrow k}$,
avremo che $G=F_{k-1}^\sigma<F_i^\tau$, che dimostra \localref{2}.\\
Per mostrare \localref{3}, notiamo che per \localref{commutano} i sottogruppi $\Gamma_{-1\rightarrow k}$ e $\Gamma_{k-1\rightarrow n}$ commutano
puntualmente, per \localref{generano} essi generano $\Gamma$ e infine per \ref{prop:IntersectionProperty}
$\Gamma_{-1\rightarrow k}\cap\Gamma_{k-1\rightarrow n}=\{1\}$. Pertanto $\Gamma$ risulta somma diretta di tali sottogruppi, cio\`e \localref{3}.\\
\implication{2}{1}Supponiamo adesso che valga \localref{2}, ossia che ogni $(k-1)$-faccia di $\p$ sia incidente con ogni sua $k$-faccia. Per
mostrare che $p_k=2$ consideriamo una qualunque sezione $\mathcal{S}=F/G$ dove $rF=k+1$ e $rG=k-2$. Essendo $p_k=\left|\mathcal{S}_1\right|$ il numero
dei lati di $\mathcal{S}$, dimostriamo che $\mathcal{S}$ \`e un \emph{diagono} ossia un segmento doppio. Ogni vertice $G\in\mathcal{S}_0\subseteq\p_{k-1}$
\`e incidente a ogni lato $F\in\mathcal{S}_1\subseteq\p_k$.
Scelto quindi un $\overline{G}\in\mathcal{S}_0$, la sezione $\mathcal{S}/{\overline{G}}$ avr\`a come vertici tutti e soli i lati di $\mathcal{S}$ quindi
\begin{equation*}
\left|(\mathcal{S}/{\overline{G}})_0\right|=\left|\mathcal{S}_1\right|
\end{equation*}
ma - ora - ricordiamo che per la diamond property la $1$-sezione $\mathcal{S}/{\overline{G}}$ deve avere precisamente $2$ vertici, pertanto concludiamo che
\begin{equation*}
p_2=\left|\mathcal{S}_1\right|=\left|(\mathcal{S}/{\overline{G}})_0\right|=2
\end{equation*}
Cio\`e \localref{3}.
\end{proof}

\section{Quozienti}
Cominciamo a sviluppare gli strumenti che servono per definire il quoziente di un politopo. Questo verr\`a definito quozientando rispetto all'azione di
determinati sottogruppi di simmetrie del politopo in questione, per avere un'identificazione di facce che sia compatibile con la struttura di politopo.
Cionondimeno sar\`a necessario - in una prima parte - indebolire il concetto di politopo non richiedendo pi\`u la \emph{forte connessione} ottendendo
quindi una struttura che prende il nome di \emph{prepolitopo}. In questo contesto dovremo rafforzare le richieste sui morfismi tra politopi
definendo cos\`i le \emph{rap-map}.
\subsection{Prepolitopi}
Cominciamo quindi con le definizioni del caso circa i prepolitopi e le rap-map.
\begin{defin}[Prepolitopo]
Sia $\p$ un poset che soddisfa le propriet\`a $(P1)$, $(P2)$ e $(P4)$ (i tre assiomi che, insieme a $(P3)$, definirebbero un politopo). Allora diremo
$\p$ un prepolitopo di rango $n$ dato dall'assioma $(P2)$.
\end{defin}
Notiamo come i concetto di bandiera, rango, adiacenza e sezione continuino ad avere senso e ad essere definiti allo stesso modo per un prepolitopo.
Introduciamo ora le pi\`u importanti mappe nel contesto dei prepolitopi: le cosiddette \emph{rap-map}. L'acronimo sta per \emph{rank adjacency preserving}.
Infatti tali particolari morfismi preservano il rango delle facce e la relazione di adiacenza tra bandiere nei prepolitopi.
\begin{defin}
Siano $\p$ e $\mathcal{Q}$ due prepolitopi dello stesso rango $n$. Allora un omomorfismo $\phi:\p\longrightarrow\mathcal{Q}$ che soddisfi:
\begin{enumerate}
\item $rk(\phi(F))=rk(F)$ per ogni $F\in\p$ e
\item se $\Phi,\Psi\in\flag{\p}$ sono incidenti allora lo sono anche $\phi(\Phi)$ e $\phi(\Psi)$ in $\mathcal{Q}$
\end{enumerate}
\`e detto una \emph{rap-map}.
\end{defin}
Per definire i quozienti sar\`a chiaramente importante il caso in cui una rap-map sar\`a suriettiva.
\begin{defin}
Siano $\p$ e $\mathcal{Q}$ due prepolitopi (dello stesso rango) e $\phi:\p\longrightarrow\mathcal{Q}$ una rap-map. Allora, se $\phi$ \`e suriettiva
diremo $\phi$ ricoprimento e altres\`i che $\p$ ricopre $\mathcal{Q}$.\\
Inoltre, nel caso in cui $\phi$ mandi isomorficamente sezioni di $\p$ di rango $k$ in sezioni (di rango $k$) di $\mathcal{Q}$, diremo che
$\phi$ \`e un $k$-ricoprimento.
\end{defin}
Osserviamo che particolarmente interessante sar\`a per $n$-prepolitopi il caso degli $(n-1)$-ricoprimenti, in quanto questi
preservano la struttura delle figure al vertice e delle facce.\\
Cominciamo quindi a vedere come si comportino queste mappe con un primo risultato.
\begin{lem}
\label{lem:Connectness-Covering}
Siano $\p$ e $\mathcal{Q}$ due $n$-prepolitopi e $\phi:\p\longrightarrow\mathcal{Q}$ una rap-map. Se $\mathcal{Q}$ \`e connesso per bandiere allora
$\phi$ \`e un ricoprimento.
\end{lem}
\begin{proof}
Dobbiamo mostrare che sotto le ipotesi, $\phi$ risulta suriettiva. Sia pertanto $G\in\mathcal{G}$. Scelta $\Phi$ una bandiera di $\p$, notiamo che,
come nel caso di politopi, essa ha tipo $(-1,\cdots,n)$ ed, essendo $\phi$ una rap-map, $\phi(\Phi)$ ha il medesimo tipo ed \`e pertanto una bandiera
di $\mathcal{Q}$. Per sfruttare l'ipotesi di connessione (per bandiere) di $\mathcal{Q}1$ estendiamo $\{G\}$ a una bandiera
$\Psi\in\flag{\mathcal{G}}$. Avremo quindi che esisteranno ranghi $-1<r_0\leq\cdots\leq r_l<n$ tali che
\begin{equation*}
\Psi=\phi(\Phi)^{r_0\cdots r_l}=\phi(\Phi^{r_0\cdots r_l})=:\phi(\Phi')
\end{equation*}
pertanto $G\in\Psi=\phi(\Phi')\subseteq\phi(\p)$ da cui la suriettivit\`a.
\end{proof}
Essendo interessati principalmente ai politopi vediamo una propriet\`a interessante che ci garantisce che l'immagine di un politopo tramite
una rap-map sia anch'essa un politopo (quindi sostanzialmente che sia fortemente connesso, ossia $(P4)$).\\
Una condizione necessaria e sufficiente \`e che la mappa sia "localmente suriettiva'' in un certo senso, pi\`u formalmente che mandi sezioni
del dominio in sezioni del codominio. Vediamo quindi la seguente:
\begin{prop}
\label{prop:PrepolytopeCovering}
Siano $\p$ un politopo e $\mathcal{Q}$ un prepolitopo, entrambi di rango $n$, e $\phi:\p\longrightarrow\mathcal{Q}$ una rap-map. Allora sono equivalenti:
\begin{itemize}
\item$\mathcal{Q}$ \`e un politopo
\item$\phi(F/G)=\phi(F)/\phi(G)$ per ogni $G\leq F\in\p$
\end{itemize}
Inoltre, quando ci\`o accade, $\mathcal{Q}$ risulta ricoperto da $\p$ a mezzo di $\phi$.
\end{prop}
\begin{proof}
\implication{1}{2}Supponiamo dapprima che $\mathcal{Q}$ sia un politopo. Sia $F/G$ una sezione di $\p$, vogliamo mostrare che anche $\phi(F/G)$
risulta sezione (di $\mathcal{Q}$), ossia che $\phi(F/G)=\phi(F)/\phi(G)$.\\
Essendo $\p$ e $\mathcal{Q}$ entrambi politopi abbiamo che le due sezioni $F/G$ e $\phi(F)/\phi(G)$ sono politopi e che
\begin{equation*}
\phi\restrict{F/G}:F/G\longrightarrow\phi(F)/\phi(G)\subseteq\mathcal{Q}
\end{equation*}
risulta ancora una rap-map. Inoltre notiamo che - per la forte connessione di $\mathcal{Q}$ - la sezione $\phi(F)/\phi(G)$ \`e connessa, da cui,
applicando \ref{lem:Connectness-Covering}, abbiamo che $\phi\restrict{F/G}$ \`e suriettiva su $\phi(F)/\phi(G)$, cio\`e quanto si voleva mostrare.\\
\implication{2}{1}Supponiamo quindi che l'immagine di ogni sezione di $\p$ sia una sezione di $\mathcal{Q}$. Mostriamo intanto che $\phi$ risulta
un ricoprimento. Ricordando che $\phi$ preserva rango e condizione d'adiacenza, la sua suriettivit\`a discende dalla seguente considerazione:
\begin{equation*}
\phi(\p)=\phi(\p/\bot_\p)=\phi(\p)/\phi(\bot_\p)=\mathcal{Q}/\bot_\mathcal{Q}=\mathcal{Q}
\end{equation*}
Mostriamo ora che se $H/K$ \`e una sezione di $\mathcal{Q}$, allora questa risulta connessa. Per la suriettivit\`a di $\phi$ appena mostrata
esiste $F\in\p$ tale che $\phi(F)=H$ ma ogni faccia sotto $H$ \`e immagine di una faccia sotto $F$,
in quanto $\phi(F/\bot_\p)=\phi(F)/\bot_\mathcal{Q}=H/\bot_\mathcal{Q}$, pertanto sia $G\leq F\in\p$ tale che $\phi(G)=K$.\\
Siano $L',L''\in H/K$, mostriamo che queste sono connesse da una catena di facce (proprie) in $H/K$. Per ipotesi abbiamo che
$\phi(F)/\phi(G)=H/K$ pertanto esisteranno $T',T''\in F/G$ tali che $\phi(T')=L'$ e $\phi(T'')=L''$, per la connessione di $F/G$ avremo inoltre che
esiste una successione di facce proprie $T'=S_0,\cdots,S_k=T''$ che connette $T'$ a $T''$. Allora l'immagine della successione, cio\`e
$L'=\phi(S_0),\cdots,\phi(S_k)=L''$ sar\`a formata da facce proprie ($\phi$ preserva il rango delle facce) di $H/K$ che connettono
($phi$ essendo un omomorfismo rispetta l'adiacenza tra le facce) $L'$ a $L''$, da cui connessione di $H/K$ e quindi per la genericit\`a della scelta
della sezione in $\mathcal{Q}$, la forte connessione di $\mathcal{Q}$.

La proposizione \ref{prop:PrepolytopeCovering} appena vista sostanzialmente ci d\`a una condizione necessaria e sufficiente affinch\`e l'immagine di
un politopo (basta chiedere che sia semplicemente codominio) tramite una rap-map sia anch'essa un politopo. In effetti tale condizione esprime
una \emph{locale suriettivit\`a} della mappa. Ossia una suriettivit\`a sulle sezioni. Riformuliamo infatti la sopracitata proposizione con un suo
corollario.
\begin{corol}
\label{corol:PrepolytopeCovering}
Siano $\p$ politopo, $\mathcal{Q}$ un prepolitopo e $\phi:\p\longrightarrow\mathcal{Q}$ una rap-map. Allora $\mathcal{Q}$ risulta un politopo
se e solo se per ogni $G\leq F\in\p$ e per ogni $\phi(G)\leq\tilde{H}\leq\phi(F)$ esiste una $G\leq H\leq F$ tale che ${H}=\phi(H)$
\end{corol}


\end{proof}
\end{document}


% Dire che F e 0 sono massimo e minimo risp.
% Cambiare notazione, tipo x\in P? Lo preferirei molto in verit\`a
% Cambiare altre notazioni, fare un TODO, aggiornare tutto con \flag
% · Classificare i poligoni
% · Corrispondenza di Galois
